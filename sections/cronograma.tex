\section{Cronograma de Execução}
	\label{sec:cronograma}

Para explorar a metodologia utilizada as atividades a serem realizadas foram
definidas e detalhadas segundo seu período de execução, conforme a Tabela
\ref{tab:cronograma}.

\begin{enumerate}

\item Estudo, levantamento bibliográfico de trabalhos correlatos para escolha de
técnicas de agrupamento, redução de dados, etc.;

\item Definição da motivação do problema e levantamento sobre bases de dados
para avaliação do trabalho desenvolvido;

\item Desenvolvimento de uma abordagem para o agrupamento hierárquico
aglomerativo de séries espaço-temporais, incluindo uma etapa de
pré-processamento baseada em redução de dados;

\item Desenvolvimento de uma abordagem para a geração de amostras de séries
espaço-temporais por meio da análise de auto-similaridade;

\item Avaliação experimental das novas abordagens em termos de tempo de
execução, consumo de memória e qualidade dos agrupamentos gerados visando o
agrupamento dos diversos tipos de vegetação para identificação de áreas de
plantio de culturas como a cana-de-açúcar.

\item Preparação de artigo científico e da dissertação de mestrado

\item Defesa da dissertação

\end{enumerate}  


\begin{table}[htbp]
\centering
\setlength{\tabcolsep}{0pt}
\begin{tabular}{|c|c|c|c|c|c|c|c|c|c|c|c|c|c|c|}
\hline
\multirow{2}{*}{ \ Atividades \ } & \multicolumn{6}{c|}{2015}               & \multicolumn{8}{c|}{2016} \\ \cline{2-15} 
                                  & Jul. & Ago. & Set. & Out. & Nov. & Dez. & Jan. & Fev. & Mar. & Abr. & Mai. & Jun. & Jul. & Ago. \\ \hline
1                                 & \x   & \x   & \x   &      &      &      &      &      &      &      &      &      &      &      \\ \hline
2                                 & \x   & \x   & \x   &      &      &      &      &      &      &      &      &      &      &     	\\ \hline
3                                 &      &      &      & \x   & \x   &      &      &      &      &      &      &      &      &     	\\ \hline
4                                 &      &      &      &      &      & \x   & \x   &      &      &      &      &      &      &     	\\ \hline
5                                 &      &      &      &      &      &      &      & \x   & \x   & \x   &      &      &      &     	\\ \hline
6                                 &      &      &      &      &      &      &      &      &      &      & \x   & \x   & \x   &     	\\ \hline
7                                 &      &      &      &      &      &      &      &      &      &      &      &      &      & \x  	\\ \hline
\end{tabular}
\caption{Cronograma de execução}
\label{tab:cronograma}
\end{table}