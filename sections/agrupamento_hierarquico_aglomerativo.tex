\section{Agrupamento hierárquico aglomerativo de séries espaço-temporais}
	\label{sec:agrupamento_hierarquico}

Nesta seção será descrita a técnica proposta para agrupamento hierárquico de
séries-temporais. Basicamente, esta técnica consiste em reduzir o tamanho do
conjunto de dados por meio de técnicas de amostragem, aplicar o agrupamento
aglomerativo, e finalmente, atribuir as instâncias restantes aos seus grupos
mais próximos, como mostra o Algoritmo \ref{alg:proposta}.

\begin{algorithm}
	\Entrada{Conjunto de dados $D$, número de grupos $K$}
	\Saida{Agrupamento $\mathcal{C}$}
	1. Selecionar amostra $\mathcal{D}$,
		tal que $\left|\mathcal{D}\right| < \left|D\right|$ \;
		
	2. Obter o agrupamento hierárquico
		$\mathcal{H} \gets AGNES\left(\mathcal{D}\right)$ \;
		
	3. Aplicar o procedimento de poda em $\mathcal{H}$, obtendo
		o agrupamento $\mathcal{C}_0$, com $K$ grupos \;
		
	4. Atribuir os objetos restantes em $D \setminus \mathcal{D}$
		aos grupos em $\mathcal{C}_0$, obtendo o agrupamento final $\mathcal{C}$ \;
	
	\caption{Agrupamento Hierárquico Aglomerativo com Amostragem}
	\label{alg:proposta}
\end{algorithm}

A redução de dados tem papel fundamental na obtenção da escalabilidade dos
algoritmos de agrupamento. Assim, o verdadeiro desafio na aplicação das técnicas
de redução de dados é manter as características do conjunto de dados original,
para que seja possível descobrir a estrutura dos grupos presentes.

A redução do tamanho do conjunto de dados por meio de amostragem aleatória é uma
técnica utilizada em vários algoritmos de agrupamento, entre eles CURE
\cite{guha1998cure}, CLARA \cite{kaufman1990finding} e YADING \cite{Ding2015}.
Nestes trabalhos, a amostragem aleatória é aplicada estimando-se o tamanho
mínimo das amostras para que as características dos grupos sejam preservadas.

\subsection{Amostragem Baseada em Fractais}
	\label{subsec:reducao_fractais}

Em vários trabalhos da literatura científica da área a análise de fractais
mostrou-se uma técnica promissora na redução da dimensionalidade dos dados.
No entanto, até o momento, não foram encontrados trabalhos que tenham aplicado
técnicas de análise de fractais à redução de conjuntos de dados
espaço-temporais.

Neste trabalho propõe-se a aplicação da técnica de \emph{Box-plot counting} para
efetuar a redução de dados. Esta técnica permite detectar a auto-similaridade
dos dados pela contagem das células que contêm um ou mais pontos. Essa
característica será usada para a detecção de ruídos e \emph{outliers},
permitindo a escolha de amostras estratificadas pela densidade dos 
hiper-retângulos em vários níveis de resolução.