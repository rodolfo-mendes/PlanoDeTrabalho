\section{Introdução} \label{sec:introducao}

A análise de séries temporais de imagens de satélite tem se mostrado uma
importante ferramenta no estudo dos impactos das mudanças climáticas globais.
A partir destas séries temporais, é possível analisar o uso e a cobertura do
solo em uma determinada região, e como este uso varia ao longo do tempo.

Uma das possíveis técnicas para extrair conhecimento destas imagens é o
agrupamento de séries temporais. A partir das imagens de satélite, métricas como
temperatura da superfície, índice de reflexão e índice de vegetação são
extraídos para cada pixel da imagem, para cada instante no tempo. O agrupamento 
das séries temporais obtidas, combinado com a visualização em mapa a quais
grupos cada série pertence, permite ao tomador de decisão entender como o uso do
solo está distribuído em uma região.

Por sua vez, o avanço das tecnologias de sensoreamento têm permitido que
satélites coletem imagens com resoluções cada vez maiores. Isso significa que os
conjuntos de dados são representados por um número cada vez maior de séries
temporais, na ordem de milhões e até mesmo bilhões de séries para serem
agrupadas.

Por meio do algoritmo k-médias é possível agrupar dados com complexidade de
tempo linear, $O(n)$. Porém, esse algoritmo possui algumas desvantagens com
relação ao significado dos agrupamentos gerados. Uma destas desvantagens é a
necessidade do usuário fornecer antecipadamente o número de agrupamentos
desejados através do parâmetro $k$. Ou seja, é necessário que o usuário tenha
algum conhecimento prévio sobre o domínio do problema e que ele tenha uma
estimativa do número de grupos existente no conjunto de dados. Outra limitação
está relacionada ao formato dos agrupamentos no espaço dos atributos.
No k-médias, cada instância do conjunto de dados é atribuída ao grupo cujo
centro esteja mais próximo, que por sua vez é recalculado a cada iteração.
Assim, os agrupamentos tendem a ter formatos esféricos ou regiões densas são
divididas por hiperplanos separadores projetados, o que nem sempre corresponde à
melhor descrição dos dados.

Uma abordagem alternativa são os algoritmos hierárquicos aglomerativos. Nestes
algoritmos, cada instância do conjunto de dados é colocada em seu próprio grupo
inicialmente. Em seguida, os agrupamentos são aglutinados em grupos maiores, 
formando uma hierarquia de grupos. Diferentemente do k-médias, o usuário
não precisa fornecer o número $k$ de grupos previamente. Embora a saída do 
algoritmo hierárquico seja um dendograma (uma hierarquia de grupos), este pode
ser convertido em uma partição de $k$ grupos por meio de uma poda na árvore
resultante. Por sua vez, as partições resultantes deste processo não estão
limitadas a formatos esféricos, sendo o algoritmo capaz de detectar agrupamento
com formatos arbitrários. Apesar dessas vantagens, os algoritmos hierárquicos
tem alto custo computacional, de ordem $O(n^2)$ (quadrática) até $O(n^3)$
(cúbica) para tempo e espaço, tornando sua aplicação impraticável para grandes
conjuntos de dados como séries temporais de imagens de satélites de alta
resolução.

Neste trabalho, propomos uma nova abordagem para que se possa aproveitar
as vantagens dos algoritmos hierárquicos mesmo em grandes conjuntos de dados.
Esta abordagem consiste em, inicialmente, reduzir o conjunto de dados, e em
seguida, aplicar o agrupamento hierárquico neste conjunto de dados reduzido. Por
fim, as instâncias restantes são aglomeradas ao grupo mais próximo, utilizando a
mesma medida de distância entre grupos utilizada no agrupamento inicial.
Espera-se que, com esta abordagem, seja possível aplicar o agrupamento
hierárquico em tempo consideravelmente menor, mas que a perda de qualidade dos
agrupamentos resultantes seja mínima.

Esta proposta de trabalho está dividida da seguinte forma: na Subseção
\ref{subsec:objetivos} estão listados o objetivo geral da pesquisa e os
objetivos mais específicos. Na Subseção \ref{subsec:hipotese} encontra-se a 
formulação da hipótese a ser averiguada durante o trabalho e na Subseção
\ref{subsec:contribuicao} destaca-se a contribuição do trabalho. Na Seção
\ref{sec:fundamentacao_teorica}, são apresentados os principais conceitos
teóricos necessários para o desenvolvimento e entendimento do trabalho. Na Seção
\ref{sec:agrupamento_hierarquico} é apresentada a proposta de algoritmo
hierárquico aglomerativo baseada em redução de dados. Na Seção
\ref{sec:resultados_preliminares} são apresentados os resultados preliminares
obtidos ao aplicar a técnica proposta. Na Seção \ref{sec:metodologia},
descreve-se a metodologia que será utilizada para o prosseguimento do trabalho.
Por fim, na Seção \ref{sec:cronograma} é apresentado o cronograma para a
execução do restante da pesquisa.

\subsection{Objetivos e Desafios de Pesquisa}
	\label{subsec:objetivos}

O objetivo geral deste trabalho é desenvolver uma abordagem escalável para o
agrupamento hierárquico de séries espaço-temporais de imagens de satélite que
permita reduzir significativamente a complexidade de tempo de execução e espaço
do algoritmo, sem no entanto reduzir a qualidade dos agrupamentos produzidos.
Especificamente, deseja-se obter:

\begin{enumerate}
    \item Desenvolvimento de uma abordagem para o agrupamento hierárquico
		aglomerativo de séries espaço-temporais incluindo uma etapa de
		pré-processamento baseada em redução de dados;

    \item Desenvolvimento de uma abordagem para a geração de amostras de séries
		espaço-temporais por meio da análise de auto-similaridade;	
    
    \item Avaliar experimentalmente as novas abordagens em termos de tempo de
		execução, consumo de memória e qualidade dos agrupamentos gerados visando o
		agrupamento dos diversos tipos de vegetação para identificação de áreas de
		plantio de culturas como a cana-de-açúcar.
\end{enumerate}

A etapa 1 encontra-se concluída e os resultados experimentais iniciais
encontram-se descritos na Seção \ref{sec:resultados_preliminares}. A
implementação da etapa 2 está em andamento espera-se concluí-la em .... 
	
\subsection{Hipótese}
	\label{subsec:hipotese}

A hipótese deste projeto de pesquisa consiste em: "A redução do conjunto dos
dados de entrada permite executar o agrupamento hierárquico aglomerativo em
tempo menor, mas mantendo a qualidade dos agrupamentos produzidos".

\subsubsection{Amostragem baseada em fractais}

A hipótese de uso da teoria dos fractais para a geração de amostras
significativas baseadas na análise da auto-similaridade vem do fato de que
técnicas de amostragem ingênuas (\textit{naive}), como amostragens aleatórias e
amostragens estratificadas, não são adequadas para conjuntos de dados reais com
presença de dados ruidosos e de exceções, uma vez que o comportamento desses
algoritmos é imprevisível.
Como não há atributos de classe disponíveis, pretende-se fazer uma amostragem
baseada na densidade da estrutura \textit{box count}. O objetivo é obter um
conjunto mínimo de instâncias que represente as características do conjunto
original.

\subsection{Contribuição}
	\label{subsec:contribuicao}

A contribuição esperada é uma estratégia para o agrupamento hierárquico
aglomerativo que produza respostas mais rápidas para o usuário.

