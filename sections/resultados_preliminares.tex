\section{Resultados Preliminares}
	\label{sec:resultados_preliminares}

Nesta seção serão descritos os experimentos preliminares cujo intuito foi
comparar o agrupamento hierárquico por amostragem, proposto na seção
\ref{sec:agrupamento_hierarquico} e o algoritmo AGNES, que constrói o
agrupamento hierárquico aglomerativo com todo o conjunto de dados. Os resultados
tabelados podem ser encontrados no Apêndice.

O conjunto de dados utilizado nos testes são valores de índice NDVI
(\emph{Normalized Difference Vegetation Index}) de uma área localizada entre
as latitudes $-8,55$ e $-8,45$, e as longitudes $-38,25$ e $37,25$, que 
corresponde a uma região do estado de Pernambuco, Brasil. Estes dados foram
extraídos a partir de imagens coletadas por um satélite MODIS
(\emph{Moderate Resolution Imaging Spetroradiometer}) durante o ano de 2003,
em períodos de 16 dias. Assim, a cada \emph{pixel} da imagem é associada uma
série temporal composta pelos valores do índice NDVI coletados ao longo do ano.
O conjunto de dados utilizado possui um total de $9812$ séries temporais, cada
uma $23$ coletas do índice NDVI. A tabela \ref{tab:amostra_dados} mostra um
subconjunto destes dados.

% latex table generated in R 3.2.4 by xtable 1.8-2 package
% Mon Jun 13 22:43:15 2016
\begin{table}[htbp]
\centering
\begin{tabular}{|c|c|r|r|r|}
  \hline
	\textbf{Latitude} & \textbf{Longitude} & \textbf{01/01/2003} & \textbf{17/01/2003} & \textbf{02/02/2003} \\ 
  \hhline{|=|=|=|=|=|}
	-8.36 & -40.89 & 0.76 & 0.83 & 0.76 \\ \hline
  -8.36 & -40.89 & 0.71 & 0.83 & 0.76 \\ \hline
  -8.36 & -40.89 & 0.73 & 0.81 & 0.62 \\ \hline
  -8.36 & -40.89 & 0.68 & 0.74 & 0.72 \\ \hline
  -8.36 & -40.89 & 0.69 & 0.74 & 0.72 \\
	\hline
\end{tabular}
\caption{Amostras do conjunto de dados MODIS-PE-2003.} 
\label{tab:amostra_dados}
\end{table}


O experimento consistiu em comparar o algoritmo hierárquico aglomerativo por
amostragem com o algoritmo AGNES, ambos utilizando a ligação simples
(\emph{single linkage}) como medida de distância entre grupos. Os algoritmos
foram comparados em relação ao seu tempo de execução e à qualidade dos
agrupamentos gerados. Para medir a qualidade dos agrupamentos, utilizou-se
o índice Dunn e o índice Davies-Bouldin.

Para cada um dos algoritmos comparados, foi realizada a poda da hierarquia
gerada em grupos $k = 3,5,7$. Por ser determinístico, o algoritmo AGNES foi
executado uma única vez para cada valor de $k$. Já o algoritmo por amostragem
foi executado com amostras de tamanho $m = 10,100,1000$. Por sua natureza 
probabilística, o algoritmo por amostragem foi executado $10$ vezes para cada
combinação de $m$ e $k$, e foram calculadas as médias do tempo de execução
e das métricas de qualidade dos agrupamentos. O gráfico da Figura 
\ref{fig:tempo_execucao} mostra a comparação dos tempos de execução em escala
logarítmica (a amostra de tamanho 9812 corresponde à execução do algoritmo
AGNES).

\figura{images/executionTime.pdf}{0.6}
{Tempo de execução por tamanho de amostra}
{fig:tempo_execucao}

Embora o algoritmo baseado em amostragem precise, ao final do algoritmo,
atribuir as instâncias restantes ao grupo mais próximo, o tempo de execução 
dessa etapa é da ordem de $O\left(nm\right)$, ou seja, linear em relação ao 
tamanho $n$ do conjunto de dados. Assim, diminuir o número de instâncias
processadas pelo algoritmo aglomerativo representou uma diminuição significativa
no tempo de execução do agrupamento, como esperado. 

Para comparar a qualidade dos agrupamentos produzidos foram utilizados os
índices Dunn e Davies-Boulding. Além do tamanho da amostragem, também foi
avaliado se o número de grupos utilizados na poda influenciaria a qualidade dos
agrupamentos produzidos.

O índice Dunn é obtido pela razão entre a menor distância entre pares de grupos
diferentes e a maior distância entre pares de um mesmo grupo. Assim, para
agrupamentos de maior qualidade, o valor desse índice é maior. Como pode ser 
observado no gráfico da Figura \ref{fig:indice_dunn}, o número de grupos
utilizado na poda influenciou pouco a qualidade dos agrupamentos obtidos. Por
sua vez, o fator determinante na qualidade foi o tamanho da amostra utilizada
nos algoritmos.

\figura{images/indiceDunn.pdf}{0.6}
{Índice Dunn por tamanho da amostra e número de grupos}
{fig:indice_dunn}

Houve grande diferença entre a qualidade dos agrupamentos produzidos pelo
algoritmo AGNES e os agrupamentos produzidos pelo algoritmo baseado em
amostragem. Dado que o índice Dunn é influenciado pela menor distância entre
pares de grupos diferentes, esse desempenho pode ser explicado pelo fato do 
algoritmo AGNES garantir que os pares de objetos mais próximos serão colocados 
no mesmo grupo. Por outro lado, no algoritmo baseado em amostragem, existe a
possibilidade de que objetos muito próximos sejam colocados em grupos separados,
afetando negativamente seu desempenho.

A qualidade inferior dos agrupamentos produzidos pelo algoritmo baseado em 
amostragem também foi refletida no índice Davies-Boulding. Este índice é obtido
para cada par de grupos como a relação entre a soma dos desvios-padrão e a 
distância entre as médias dos grupos. Assim, quanto maior a qualidade dos 
agrupamentos, menor será o valor desse índice, pois menor será a dispersão
dentro de um grupo e maior será a distância entre seus centros. O gráfico da
Figura \ref{tab:indice_db} mostra os resultados obtidos.

\figura{images/indiceDaviesBouldin.pdf}{0.6}
{Índice Davies-Bouldin por K e tamanho de amostra}
{fig:indice_davies_bouldin}

Os dados mostrados no gráfico da Figura \ref{tab:indice_db} também indicam que a
qualidade dos agrupamentos produzidos pelo algoritmo baseado em amostragem foi
inferior. Isso também pode ser explicado pela garantia que o algoritmo AGNES
oferece de agrupar os pares mais próximos já no primeiro passo do algoritmo,
pois diminui a dispersão dos dados. O gráfico da Figura
\ref{fig:indice_davies_bouldin} mostra que, com relação ao índice Dunn, este
índice sofreu maior variação de acordo com o número de grupos e o tamanho das
amostras.

\subsection{Conclusões Preliminares}
	\label{subsec:conclusoes_preliminares}
	
Por meio destes experimentos confirmou-se que o algoritmo proposto na Seção
\ref{sec:agrupamento_hierarquico} permite que o agrupamento hierárquico
aglomerativo seja executado em tempo muito menor que o algoritmo original.
Porém, a amostragem aleatória impactou negativamente na qualidade dos
agrupamentos obtidos.

No entanto, ainda não é possível descartar a utilização das técnicas de redução
de dados para obter melhor desempenho do agrupamento hierárquico aglomerativo.
Ao longo do trabalho proposto, pretende-se ampliar o escopo dos experimentos,
averiguando o comportamento de outras medidas de distância, variar o tamanho
das amostras e o número de grupos, assim como implementar novos métodos de 
redução de dados.
