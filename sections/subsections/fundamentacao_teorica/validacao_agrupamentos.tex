\subsection{Validação de Agrupamentos}
	\label{subsec:validacao_agrupamentos}
	
A validação de agrupamentos é uma tarefa de pós-processamento da descoberta de
conhecimento em bancos de dados. Dadas as diferentes abordagens para se produzir
agrupamentos, e a variabilidade dos parâmetros necessários a esses algoritmos,
surge a necessidade de se estabelecer critérios objetivos para avaliar e 
comparar os resultados dos algoritmos de agrupamento. Existem diversas métricas
propostas na literatura para validar a qualidade de agrupamentos. Elas são 
dividas basicamente em dois grupos:

\begin{itemize}
	\item \textbf{Externas}: o agrupamento é avaliado segundo critérios externos
	ao agrupamento. Um exemplo seria o conhecimento prévio sobre a categoria de
	cada uma das instâncias do conjunto de dados;
	
	\item \textbf{Internas}: neste caso, o agrupamento é avaliado a partir de 
	características inerentes ao próprio agrupamento resultante. Duas medidas 
	bastante comuns são a coesão e a separação dos agrupamentos. 
\end{itemize}

Nas próximas sessões, serão apresentadas as principais métricas que poderão
ser utilizadas neste trabalho de pesquisa.

\subsubsection{Pureza}
	\label{subsubsec:pureza}

A pureza é uma medição externa da qualidade de um agrupamento. Para medi-la, é
necessário que a cada objeto do conjunto de dados esteja associada uma
categoria. A pureza mede a uniformidade dos grupos obtidos. Quanto menor o
número de categorias presentes uma cada grupo, maior será a pureza do
agrupamento. Para um determinado grupo $C_i$, a sua pureza é calculada pela
Equação \ref{eq:pureza_grupo}:

\begin{equation}
	pureza_i = \frac{1}{n_i} \max_{j = 1}^{k}{\left\{n_{ij}\right\}},
	\label{eq:pureza_grupo}
\end{equation}

\noindent onde $n_i$ é o número de objetos no grupo $C_i$ e $n_{ij}$ é o número
de objetos pertencentes à categoria $j$ que estão presentes no grupo $C_i$. Por
sua vez, a pureza do agrupamento é obtida pela média ponderada das purezas de 
cada um dos $r$ grupos, como mostra a Equação \ref{eq:pureza_agrupamento}.

\begin{equation}
	pureza = \sum_{i=1}^{r}{\frac{n_i}{n} pureza_i}
	\label{eq:pureza_agrupamento}
\end{equation}

A pureza de um agrupamento pode assumir valores entre $0$ e $1$, sendo que o
valor $1$ indica o grau máximo de pureza para um agrupamento. Destaca-se o fato
de que é possível um agrupamento obter $pureza = 1$ mesmo quando o número de 
grupos é maior que o número de categorias. Nesse caso, os grupos são
subconjuntos de cada categoria.

\subsubsection{Índice Dunn}
	\label{subsubsec:indice_dunn}
	
O índice Dunn é uma medida interna da qualidade de um agrupamento. Ele mede a 
qualidade do agrupamento como a razão entre a separação entre os grupos e a
coesão dos objetos de um mesmo grupo. Seu valor pode variar entre $0$ e
$+\infty$, de forma que quanto maior o valor do índice Dunn, melhor será a
qualidade do agrupamento.

Este índice é baseado em uma visão de grafos dos agrupamentos. A separação entre
os grupos é representada pela menor distância entre objetos de grupos
diferentes, enquanto que a coesão é medida pelo maior diâmetro de um grupo, como
é mostrado pelas respectivas distâncias $A$ e $B$ na Figura
\ref{fig:distancias_dunn}.

\figura{images/ilustrationDunn.pdf}{0.6}
{Separação e coesão de grupos no cálculo do índice Dunn.}
{fig:distancias_dunn}

\subsubsection{Índice Davies-Boulding}
	\label{subsubsec:indice_davies_boulding}
	
O índice Davies-Boulding é também uma medida interna da qualidade de um
agrupamento. Assim como o índice Dunn, o índice Davies-Boulding reúne em uma
grandeza escalar tanto a coesão quanto a separação de um agrupamento. No
entanto, este índice é baseado em uma representação de protótipos dos
agrupamentos.

Seja $\mu_i$ o centroide de um grupo, denotado pela Equação \ref{eq:centroid}. 
O desvio-padrão desse grupo é obtido pela Equação \ref{eq:std_deviation}.

\begin{equation}
	\sigma_i = \sqrt{
								\frac
									{
										\sum_{x \in C_i}{ \left( \left| x - \mu_i \right| \right)^2}
									}
									{n_i}
							}
	\label{eq:std_deviation}
\end{equation}

Dado um par de grupos $C_i$ e $C_j$, o valor do índice Davies-Bouldin para estes
grupos é dado pela Equação \ref{eq:indice_db_par}.

\begin{equation}
	DB_{ij} = \frac{\sigma_i + \sigma_j}{\left| \mu_i - \mu_j \right|}
	\label{eq:indice_db_par}
\end{equation}

Por fim, o valor do índice Davies-Boulding para o agrupamento todo é o valor 
máximo do índice entre os pares normalizado pelo número $k$ de grupos, como
mostra a Equação \ref{eq:indice_db_total}.

\begin{equation}
	DB = \frac{1}{k}
			 \sum_{i = 1}^{k}
			 {
					\max_{i \neq j}{\left\{ DB_{ij} \right\}}
			}
	\label{eq:indice_db_total}
\end{equation}
 

Assim, dada a representação baseada em protótipos do índice Davies-Boulding,
observa-se que a separação entre dois grupos é dada pela distância entre seus
centróides, enquanto que a coesão dos grupos é representada pela soma de seus
respectivos desvios-padrão. Assim, quanto menor o valor do índice
Davies-Boulding para um par de grupos, melhor será a qualidade do agrupamento.