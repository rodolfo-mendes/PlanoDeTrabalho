\subsection{Séries Temporais}
	\label{subsec:series_temporais}

Uma série temporal é uma sequencia ordenada de valores coletados ao longo do
tempo. Elas são usadas para medir fenômenos que variam com o tempo, e sua
análise permite a construção de modelos que explicam estes fenômenos
\cite{morettin2006analise}. São exemplos de séries temporais: o volume de
poupança em um banco ao longo dos meses, a temperatura média de uma cidade
durante a semana, ou ainda, o número de acessos a um \textit{website} durante o dia.

Uma série temporal pode ser representada por uma função $Z(t)$, cujo domínio
representa o tempo, e $t$, um determinado instante no tempo. Dada essa
representação, uma série temporal pode ser classificada em discreta ou contínua,
de acordo com o tipo de conjunto que representa o tempo
\cite{morettin2006analise}. Por exemplo: o faturamento diário de um supermercado
é uma série temporal discreta, pois o tempo é representado por um conjunto de
intervalos enumeráveis, neste caso, dias. Por outro lado, a velocidade de um
carro de corrida durante uma prova é um exemplo de série temporal contínua, pois
a duração da corrida é um intervalo contínuo de tempo. Ainda, uma série temporal
discreta pode ser obtida a partir de uma série contínua, bastando tomar amostras
dos valores da série contínua em intervalos regulares de tempo.

No entanto, há casos em que são necessárias mais de uma variável para
representar o fenômeno medido pela série temporal. Por exemplo, o
\emph{candlestick} diário de uma ação no mercado de capitais é formado por
quatro valores: preço de abertura, preço mínimo, preço máximo e preço de
fechamento. Nestes casos, a série temporal é classificada como multivariada, e é
representada por uma função vetorial
$Z\left(t\right) =
	\left[
		Z_1\left(t\right),
		Z_2\left(t\right),
		...
		Z_r\left(t\right)
	\right]$,
onde r é o número de variáveis que descrevem o fenômeno. Caso $\left(Z_t\right)$
seja um escalar, a série temporal é classificada como univariada
\cite{morettin2006analise}.