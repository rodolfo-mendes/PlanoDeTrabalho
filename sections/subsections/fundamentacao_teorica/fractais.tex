\subsection{Fractais e a propriedade de auto-similaridade}
	\label{subsec:fractais}

Um fractal pode ser definido pelo conceito de auto similaridade, no qual partes
de qualquer tamanho de um fractal são similares (exata ou estatisticamente) ao
conjunto todo. Um exemplo clássico de um fractal criado por meio da construção
repetitiva é o triângulo de Sierpinski, construído por meio de um processo
iterativo, onde se retira de um triângulo o triângulo central e para cada
triângulo resultante realiza-se o mesmo processo, recursivamente, conforme
apresentado na Figura \ref{fig:constrsierpinski}. O triângulo de Sierpinski
apresenta características interessantes, como o fato de cada triângulo interior
ser uma miniatura do triângulo em que está inserido, perímetro tendendo ao
infinito e área tendendo a zero quando o número de iterações tende ao infinito
\cite{Schroeder91}. 

\figura{images/sierpinski.png}{0.3}
{Construção do triângulo de Sierpinski \cite{Schroeder91}.}
{fig:constrsierpinski}

O conceito de auto similaridade está relacionado com periodicidade. Muitos
fenômenos naturais e humanos acontecem com periodicidade. São exemplos o uso do
solo pela agricultura, o valor das moedas e das ações, o comportamento de redes
de comunicação, as filas de caixa de supermercado, o uso das estradas, as
músicas que tocam nas rádios e os efeitos da economia na vida das pessoas. Uma
razão para a universalidade dessas movimentações harmônicas é a linearidade
aproximada de muitos sistemas e a sua invariância com deslocamento no espaço e
tempo. Experimentos realizados com alguns conjuntos de dados sintéticos e reais
mostram que os dados referentes aos fenômenos humanos caracterizam-se por
apresentarem uma distribuição fractal \cite{Traina2010}.

Para a análise da auto similaridade de conjuntos contendo fenômenos naturais e
humanos, chamados de fractais estatisticamente auto-similares, utiliza-se o
método \textit{Box-Counting}. Para encontrar o \textit{Box-Counting} de um
conjunto de dados imerso em um espaço $E$-dimensional, deve-se dividir esse
espaço em células de um hipercubo de lado $r$, recursivamente, até encontrar um
elemento por célula \cite{Traina2010}. O método foi proposto para cálculo da
dimensão fractal, que corresponde ao número mínimo de dimensões para
representação de um conjunto (dimensionalidade intrínseca). 

Ainda preciso escrever sobre box plot com base em \cite{Traina2010}.