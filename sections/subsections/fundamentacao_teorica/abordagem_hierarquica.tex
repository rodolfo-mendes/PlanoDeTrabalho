\subsection{Abordagem Hierárquica para Agrupamentos}
	\label{subsec:abordagem_hierarquica}
	
Na abordagem hierárquica de agrupamento, os grupos são organizados em árvore, de
forma que cada nodo desta árvore representa um grupo. Desta maneira,
estabelece-se uma relação pai-filho entre os grupos, tal que, dado um grupo pai
$C_p$ que tenha $n$ filhos $\{C_1, C_2,...,C_n\}$, então $C_i \subset C_p$ para
todo $1 \leq i \leq n$, como mostra a Figura \ref{fig:cluster_hierarquico}(b).
Nas folhas desta árvore, encontram-se as instâncias de dados, cada uma incluída
em seu próprio grupo. Por sua vez, na raiz encontra-se o grupo que abrange todo
o conjunto de dados. A Figura \ref{fig:cluster_hierarquico}(a) mostra a
hierarquia entre os grupos através de um diagrama conhecido como
\emph{dendrograma}.

\figura{images/cluster_hierarquico.png}{1.0}
{Exemplo de agrupamento hierárquico \cite{tan2009introducao}.}
{fig:cluster_hierarquico}

A organização de grupos em hierarquias tem aplicações em diversas áreas. Em 
recuperação da informação por exemplo, documentos podem ser agrupados de maneira
hierárquica de acordo com o assunto, revelando uma estrutura de tópicos e
sub-tópicos. Em Biologia, o agrupamento de espécies de acordo com suas
características pode ajudar na compreensão sobre como essas espécies evoluíram
ao longo do tempo.

Uma das principais vantagens da abordagem hierárquica de agrupamento é que não
é necessário que o usuário informe o número $k$ de grupos desejado previamente.
Porém, nem sempre é interessante para o usuário analisar todos os grupos obtidos
pelo agrupamento hierárquico. Nestes casos, a hierarquia original pode ser
convertida em um particionamento através do procedimento de poda
(\emph{prunning}). 

Nesta subseção serão discutidos os principais aspectos relativos ao agrupamento
hierárquicos. Serão abordados: algoritmo AGNES para agrupamento aglomerativo, 
algoritmo DIANA para agrupamento hierárquico divisivo e as principais medidas
de distância entre clusters: single linkage, complete linkage, average linkage,
mean distance e Ward's distance. Por fim serão analisados vantagens e
desvantagens da abordagem hierárquica.

\subsubsection{AGNES: algoritmo aglomerativo}
	\label{subsec:agnes}

O algoritmo AGNES (\emph{AGglomerative NESting}) é o algoritmo elementar para
executar o agrupamento hierárquico aglomerativo. Basicamente, seu procedimento
consiste alocar inicialmente cada instância de dados em seu próprio grupo e
então, sucessivamente, fundir os grupos mais próximos entre si, até que todos os
objetos sejam aglomerados em um único grupo, como mostra o Algoritmo
\ref{alg:agnes}.

\begin{algorithm}[htbp]
	\Entrada{conjunto de dados $X$}
	\Saida{agrupamento hierárquico}
	
	$n \leftarrow \left|X\right|$ \;
			
	$\mathcal{C} \leftarrow \left\{ C_i = \left\{x_i\right\} | 
		x_i \in X \right\} $ \;
		
	$M \leftarrow \left(m_{ij}\right)_{n \times n}\ |\ m_{ij} = 
		dist\left(C_i,C_j\right), C_i, C_j \in \mathcal{C}$ \;
	
	\Repita{$C \equiv X$}{
		encontra o par de grupos mais próximos $C_i$ e $C_j$ \;
		$C \leftarrow C_i \cup C_j$ \;
		$\mathcal{C} \leftarrow \left( \mathcal{C} \setminus \left\{C_i,C_j\right\}
			\right) \cup \left\{C\right\}$ \;
		recalcula $M$ \;
	}
	
	\caption{Algoritmo aglomerativo para agrupamento hierárquico}
	\label{alg:agnes}
\end{algorithm}


Após cada instância de dados ser atribuída ao seu próprio grupo, a matriz de
distâncias $M$ é calculada, armazenando a distância de cada par de grupos
existente. Então, sucessivamente, o algoritmo localiza a menor entrada na matriz
de distância $M$, que equivale a encontrar o par dos grupos mais próximos,
aglomera os grupos encontrados e recalcula a matriz de distância $M$.

Dado o algoritmo básico do agrupamento hierárquico aglomerativo, a principal
diferença entre as diferentes abordagens nesta categoria são as medidas de 
similaridade entre grupos usadas. Por sua vez, estas podem ser baseadas em
grafos ou baseadas em protótipos \cite{tan2009introducao}. São medidas de
distância baseadas em grafos:

\begin{itemize}
	\item \textbf{Ligação simples}: A ligação simples, ou \emph{single linkage},
	toma como similaridade entre dois grupos a distância entre seus elementos 
	mais próximos, e é calculada pela Equação \ref{eq:single_linkage}:
	
	\begin{equation}
		dist\left(C_i,C_j\right) =
			\min_{x_i \in C_i , x_j \in C_j}
				{\left\{ \left| x_i - x_j \right| \right\}}
		\label{eq:single_linkage}
	\end{equation}
	
	
	Ao tomar-se as instâncias de dados como vértices de um grafo, e as ligações 
	entre grupos como vértices ponderados, então o agrupamento gerado é
	correspondente a uma \emph{árvore geradora mínima} \cite{han2011data}, de
	forma que os grupos formados tendem a ser contíguos no espaço dos atributos
	\cite{tan2009introducao}.
	
	\item \textbf{Ligação completa}: A ligação completa, ou
	\emph{complete linkage} é a medida oposta à ligação simples, pois toma como 
	similaridade entre dois grupos a distância entre seus elementos mais
	distantes, sendo calculada pela Equação \ref{eq:complete_linkage}:
	
	\begin{equation}
		dist\left(C_i,C_j\right) =
			\max_{x_i \in C_i , x_j \in C_j}
				{\left\{ \left| x_i - x_j \right| \right\}}
		\label{eq:complete_linkage}
	\end{equation}
	
	
	\item \textbf{Distância Média}: Por fim, a similaridade entre dois grupos pode
	ser medida através da distância média entre os pares dos grupos
	(\emph{group average}). Essa medida é um balanceamento entre a ligação simples
	e a ligação completa, e é obtida pela média das distâncias entre cada um dos
	pares ordenados $(x_i,x_i)$, com $x_i \in C_i$ e $x_i \in C_j$. É obtida
	pela Equação \ref{eq:group_average}:
	
	\begin{equation}
		dist\left(C_i,C_j\right) =
			\frac{1}{n_in_j} 
			\sum_{x_i \in C_i}{
				\sum_{x_j \in C_j}{
					\left| x_i - x_j \right|
				}
			}
		\label{eq:group_average}
	\end{equation}
	
\end{itemize}

Além das medidas baseadas em grafos, também podem ser aplicadas medidas baseadas
em protótipos. São elas:

\begin{itemize}
	\item \textbf{Distância entre centroides}: a distância entre centroides toma
	como medida de similaridade entre grupos a distância entre as médias dos
	grupos. Seja o centroide $\mu_i$ de um grupo o seu objeto médio, dados
	pela Equação \ref{eq:centroid}:
	
	\begin{equation}
		\mu_i = \frac{1}{\left|C_i\right|} \sum_{x_i \in C_i}{x_i}
		\label{eq:centroid}
	\end{equation}
	
	Então, a distância entre os grupos é dada pela distância entre os respectivos
	centroides, como mostra a Equação \ref{eq:centroid_distance}:
	
	\begin{equation}
		dist\left(C_i,C_j\right) = \left|\mu_i - \mu_j\right|
		\label{eq:centroid_distance}
	\end{equation}
	
	\item \textbf{Método de Ward}: o método de Ward também considera o centroide
	como representante do grupo todo. Porém, a distância entre grupos é
	medida pela variação da soma dos erros quadrados, ou SSE (\emph{squared sum
	of errors}), que se obtém ao aglomerar dois grupos $C_i$ e $C_j$. Seja a soma
	dos erros quadrados de um grupo $i$, dada pela Equação \ref{eq:sse}:
	
	\begin{equation}
		SSE_i = \sum_{x \in C_i}{{\left| x - \mu_i \right|}^{2}}
		\label{eq:sse}
	\end{equation}
	
	Então, a distância de Ward será dada pela variação da soma dos erros quadrados
	obtida ao aglomerar os grupos $C_i$ e $C_j$:
	
	\begin{equation}
		dist(C_i, C_j) = \Delta SSE_{ij} = SSE_{ij} - SSE_i - SSE_j
		\label{eq:ward}
	\end{equation}
	
	Em \cite{zaki2014data}, é demonstrado que o método de Ward corresponde a uma
	versão ponderada da distância entre centroides:
	
	\begin{equation}
		dist(C_i, C_j) =
			\Delta SSE_{ij} =
			\left( \frac{n_i n_j}{n_i + n_j} \right)
				{\left| \mu_i - \mu_j  \right|}^{2}
		\label{eq:ward_weighted}
	\end{equation}
	
\end{itemize}

A primeira etapa do algoritmo AGNES envolve o cálculo da matriz de distâncias
entre os grupos iniciais. Dado o número $n$ de instâncias no conjunto de dados,
o cálculo da distância entre todos os pares de objetos envolve
$ \frac{n \left(n-1\right)}{2}$ cálculos, considerando
$dist(C_i,C_j) = dist(C_j,C_i)$. Assim, o cálculo da matriz de distâncias
tem complexidade de tempo e espaço $O\left(n^2\right)$.

Na etapa de aglomeração, o laço principal do algoritmo deve ser executado $n-1$
vezes. A cada iteração $i$, a busca pelos pares mais próximos leva tempo 
proporcional a $\left(n-i\right)^2 = O\left(n^2\right)$, assim como o recálculo
da matriz de distâncias. Portanto, o algoritmo AGNES tem complexidade de tempo
$O\left(n^3\right)$ \cite{tan2009introducao}. No entanto, com o emprego de 
estruturas de dados como o \emph{heap} é possível buscar a menor distância em 
tempo $O\left(1\right)$, enquanto que a atualização do \emph{heap} com as novas
distâncias calculadas leva $O\left(\log{n}\right)$. Portanto, a complexidade de
tempo do algoritmo nessas condições é $O\left(n^2 \log{n}\right)$
\cite{zaki2014data}.
\subsubsection{DIANA: algoritmo divisivo}
	\label{subsec:diana}

O algoritmo DIANA (DIvisive ANAlysis) é uma abordagem divisiva para o
agrupamento hierárquico \cite{kaufman1990finding}. Ao contrário da abordagem
aglomerativa, na abordagem divisiva o conjunto de dados é tomado como o grupo
inicial do agrupamento. Então, este grupo inicial é sucessivamente dividido em
grupos menores, até que cada instância de dados esteja em seu próprio grupo.
O algoritmo \ref{alg:diana} mostra este procedimento.

\begin{algorithm}[htbp]
	\Entrada{conjunto de dados D}
	\Saida{agrupamento hierárquico}
	
	$\mathcal{C} \gets \left\{ D \right\}$
	
	\Repita{$\left|C\right| = 1, \forall C \in \mathcal{C} $}{
		\ParaCada{$C \in \mathcal{C} \mid \left|C\right| > 1$}{
			$\mathcal{C} \gets \mathcal{C} \cup Divide\left(C\right)$ \;
			$\mathcal{C} \gets \mathcal{C} \setminus \left\{C\right\}$
		}
	}
	
	\caption{DIANA}
	\label{alg:diana}
\end{algorithm}

Devido aos desafios que a abordagem divisiva impõe, esta é pouco estudada na
literatura. Um dos principais problemas encontrados na abordagem divisiva é
justamente a divisão do grupo inicial. Na abordagem aglomerativa, todas as
possíveis fusões são consideradas no passo inicial, dado que, para um conjunto
de dados com $n$ objetos, existem $C_{n}^{2}$ pares possíveis, como mostra a
equação \ref{eq:agglomerative_combinations}:

\begin{equation}
	C_n^2 = \frac{n \left(n-1\right)}{2}
	\label{eq:agglomerative_combinations}
\end{equation}

Porém, para a abordagem divisiva enumerar todas as possíveis divisões de $n$
objetos em dois grupos não vazios, deverão ser consideradas $2^{n-1} - 1$
possibilidades, tornando a análise impraticável até mesmo para pequenos valores
de $n$.

Dada a impossibilidade de se analisar todas as possíveis divisões do grupo
inicial, o algoritmo DIANA utiliza uma heurística simples para dividir os
grupos. Dado um grupo $C_i$ com $n$ objetos, calcula-se a dissimilaridade de
cada objeto $x \in C_i$ através da distância média do objeto $x$ aos objetos
restantes, como mostra a Equação \ref{eq:mean_distance}

\begin{equation}
	\mu_{xi} = \frac{1}{n-1} \sum_{y \in C_i }{\left| x - y \right|}
	\label{eq:mean_distance}
\end{equation}

Para cada grupo que será dividido, mede-se a dissimilaridade de cada
objeto com relação ao resto do grupo. Então, o mais dissimilar é encontrado
e colocado em um grupo separado (\emph{splinter group}). Novamente, as
dissimilaridades são calculadas. Porém, também são calculadas as
dissimilaridades entre os objetos restantes e os objetos do grupo separado. Caso
algum objeto esteja mais próximo do grupo separado, este é transferido. Esse
processo continua até que não haja mais transferências, e ao seu final, obtém-se
a divisão em dois grupos. A Função \ref{func:divide} apresenta este procedimento.

\begin{function}[htbp]

	$ C_j \gets
		\left\{
			x \in C_{ij} \mid 
			\mu_{xij} > \mu_{yij},
			\forall y \neq x
		\right\}
	$ \;
	
	$C_i \gets C_{ij} \setminus C_j $ \;
	
	transferiu $\gets$ \textbf{true}\;
	
	\Enqto{$transferiu = \textbf{true}$}{
		transferiu $\gets$ \textbf{false} \;
		
		\ParaCada{$x \in C_i$}{
			recalcula $\mu_{xi}$ \;
			recalcula $\mu_{xj}$ \;
		}
		
		$ x \gets x \in C_{i} \mid \mu_{xi} > \mu_{yi}, \forall y \neq x$ \;
		
		\Se{$\mu_{xi} > \mu_{xj}$}{
			$C_j \gets C_j \cup \left\{x\right\}$ \;
			$C_i \gets C_i \setminus \left\{x\right\}$ \;
			transferiu $\gets$ \textbf{true} \;
		}
	}
	
	\Retorna{$\left\{C_i,C_j\right\}$}
		
	
	\caption{Divide($C_{ij}$: grupo a ser dividido)}
	\label{func:divide}
\end{function}
