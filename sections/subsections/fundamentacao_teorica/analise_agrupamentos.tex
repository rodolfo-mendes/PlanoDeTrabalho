\subsection{Análise de Agrupamentos}
	\label{subsec:analise_agrupamentos}
	
A análise de agrupamentos é uma tarefa de mineração de dados cujo objetivo é,
automaticamente, particionar o conjunto de dados em subconjuntos chamados
grupos. Os objetos reunidos em um mesmo grupo devem ser similares entre si,
enquanto que objetos de grupos separados devem ser diferentes. Ao conjunto dos
grupos resultantes da análise dá-se o nome de \emph{agrupamento}.

A análise de agrupamentos pode ser usada como uma ferramenta para extração de
conhecimento sobre um conjunto de dados ou então, como um etapa de
pré-processamento para outras tarefas de mineração de dados. Por exemplo, em
\cite{gonccalves2014land}, a análise de agrupamentos foi utilizada para
identificar o uso do terreno em diferentes regiões do estado de São Paulo,
Brasil. Já em \cite{petitjean2014dynamic}, a análise de agrupamentos foi
utilizada para eleger protótipos que posteriormente seriam utilizados como dados
de treinamento para a tarefa de classificação 1-NN.

Existem diversas abordagens para o agrupamento de dados. No agrupamento por
\emph{particionamento} o conjunto de dados é dividido em $k$ grupos, com cada 
grupo contendo pelo menos um objeto do conjunto. De maneira geral, estes
algoritmos consistem em: a partir de um agrupamento inicial, iterativamente 
realocar os objetos em grupos mais significativos até que um critério de parada
seja atingido. Podemos incluir nesta categoria os algoritmos \emph{k-médias} e
\emph{k-medoids}.

Uma abordagem alternativa é o agrupamento \emph{hierárquico}. Nesta abordagem, 
os objetos são organizados em uma hierarquia de grupos. Por sua vez, esta
hierarquia pode ser construída por duas maneiras diferentes:
\emph{aglomerativa} e \emph{divisiva}.

Na abordagem aglomerativa cada objeto de dados é inicialmente incluído em seu
próprio grupo. Em seguida, cada grupo é aglomerado com o seu grupo mais próximo,
formando uma relação "pai-filho" entre o grupo resultante e os grupos menores.
Esse processo se repete até que um único grupo, que contenha todos os dados do 
conjunto, seja obtido. Já na abordagem divisiva o processo se inverte. Todos os
objetos de dados são agrupados em um único grupo inicial, que será a raiz da 
hierarquia. Por sua vez, este grupo inicial é sucessivamente dividido em grupos
menores, até que cada objeto esteja em seu próprio grupo.

Na seção \ref{subsec:abordagem_hierarquica} a abordagem hierárquica será
explorada com mais detalhes.