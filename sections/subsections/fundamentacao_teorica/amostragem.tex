\subsection{Técnicas de Amostragem de dados}
	\label{subsec:amostragem}
	
O objetivo das técnicas de amostragem de dados é reduzir o número de instâncias
submetidas aos algoritmos de mineração de dados. Entre os desafios da amostragem
de dados estão o balanceamento das instâncias com relação à ocorrência de
instâncias raras ou de exceções. Conside um conjunto $T$ com cardinalidade
$|T| = N$. Entre as técnicas propostas na literatura destacam-se
\cite{Garcia2015}:

\begin{itemize}
    \item Amostragem aleatória de tamanho $s$ sem substituição: criada pela
		escolha de $s$ instâncias de $T$ ($s < N$), onde a probabilidade de um
		exemplo ser escolhido é de $1/N$, de modo que todas as instâncias têm a
		mesma chance de serem escolhidas;
    
    \item Amostragem aleatória de tamanho $s$ com substituição: semelhante à
		anterior, exceto pelo fato que a cada vez que uma instância é escolhida,
		permanece no conjunto e pode ser escolhida novamente;
    
    \item Amostragem balanceada: criada levando-se em consideração um conjunto
		de critérios pré-definidos, por exemplo, para manter a proporcionalidade
		de instâncias entre classes conhecidas;
    
    \item Amostragem de agrupamentos: escolha de grupos específicos resultantes
		de técnicas de agrupamento;
    
    \item Amostragem estratificada: obtida por meio da divisão de um conjunto
		$T$ em partes mutualmente disjunta seguida da escolha de uma amostragem
		aleatória em cada divisão.
\end{itemize}

% Descrever as técnicas correlatas:
% Original Data Squashing (DS) e Likelihood-based Data Squashing (LDS).