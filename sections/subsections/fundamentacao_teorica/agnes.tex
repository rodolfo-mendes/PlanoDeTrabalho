\subsubsection{AGNES: algoritmo aglomerativo}
	\label{subsec:agnes}

O algoritmo AGNES (\emph{AGglomerative NESting}) é o algoritmo elementar para
executar o agrupamento hierárquico aglomerativo. Basicamente, seu procedimento
consiste alocar inicialmente cada instância de dados em seu próprio grupo e
então, sucessivamente, fundir os grupos mais próximos entre si, até que todos os
objetos sejam aglomerados em um único grupo, como mostra o Algoritmo
\ref{alg:agnes}.

\begin{algorithm}[htbp]
	\Entrada{conjunto de dados $X$}
	\Saida{agrupamento hierárquico}
	
	$n \gets \left|X\right|$ \;
			
	$\mathcal{C} \gets \left\{ C_i = \left\{x_i\right\} | 
		x_i \in X \right\} $ \;
		
	$M \gets \left(m_{ij}\right)_{n \times n}\ |\ m_{ij} = 
		dist\left(C_i,C_j\right), C_i, C_j \in \mathcal{C}$ \;
	
	\Repita{$C \equiv X$}{
		encontra o par de grupos mais próximos $C_i$ e $C_j$ \;
		$C \gets C_i \cup C_j$ \;
		$\mathcal{C} \gets \left( \mathcal{C} \setminus \left\{C_i,C_j\right\}
			\right) \cup \left\{C\right\}$ \;
		recalcula $M$ \;
	}
	
	\caption{AGNES}
	\label{alg:agnes}
\end{algorithm}


Após cada instância de dados ser atribuída ao seu próprio grupo, a matriz de
distâncias $M$ é calculada, armazenando a distância de cada par de grupos
existente. Então, sucessivamente, o algoritmo localiza a menor entrada na matriz
de distância $M$, que equivale a encontrar o par dos grupos mais próximos,
aglomera os grupos encontrados e recalcula a matriz de distância $M$.

Dado o algoritmo básico do agrupamento hierárquico aglomerativo, a principal
diferença entre as diferentes abordagens nesta categoria são as medidas de 
similaridade entre grupos usadas. Por sua vez, estas podem ser baseadas em
grafos ou baseadas em protótipos \cite{tan2009introducao}. São medidas de
distância baseadas em grafos:

\begin{itemize}
	\item \textbf{Ligação simples}: A ligação simples, ou \emph{single linkage},
	toma como similaridade entre dois grupos a distância entre seus elementos 
	mais próximos, e é calculada pela Equação \ref{eq:single_linkage}:
	
	\begin{equation}
		dist\left(C_i,C_j\right) =
			\min_{x_i \in C_i , x_j \in C_j}
				{\left\{ \left| x_i - x_j \right| \right\}}
		\label{eq:single_linkage}
	\end{equation}
	
	
	Ao tomar-se as instâncias de dados como vértices de um grafo, e as ligações 
	entre grupos como vértices ponderados, então o agrupamento gerado é
	correspondente a uma \emph{árvore geradora mínima} \cite{han2011data}, de
	forma que os grupos formados tendem a ser contíguos no espaço dos atributos
	\cite{tan2009introducao}.
	
	\item \textbf{Ligação completa}: A ligação completa, ou
	\emph{complete linkage} é a medida oposta à ligação simples, pois toma como 
	similaridade entre dois grupos a distância entre seus elementos mais
	distantes, sendo calculada pela Equação \ref{eq:complete_linkage}:
	
	\begin{equation}
		dist\left(C_i,C_j\right) =
			\max_{x_i \in C_i , x_j \in C_j}
				{\left\{ \left| x_i - x_j \right| \right\}}
		\label{eq:complete_linkage}
	\end{equation}
	
	
	\item \textbf{Distância Média}: Por fim, a similaridade entre dois grupos pode
	ser medida através da distância média entre os pares dos grupos
	(\emph{group average}). Essa medida é um balanceamento entre a ligação simples
	e a ligação completa, e é obtida pela média das distâncias entre cada um dos
	pares ordenados $(x_i,x_i)$, com $x_i \in C_i$ e $x_i \in C_j$. É obtida
	pela Equação \ref{eq:group_average}:
	
	\begin{equation}
		dist\left(C_i,C_j\right) =
			\frac{1}{n_in_j} 
			\sum_{x_i \in C_i}{
				\sum_{x_j \in C_j}{
					\left| x_i - x_j \right|
				}
			}
		\label{eq:group_average}
	\end{equation}
	
\end{itemize}

Além das medidas baseadas em grafos, também podem ser aplicadas medidas baseadas
em protótipos. São elas:

\begin{itemize}
	\item \textbf{Distância entre centroides}: a distância entre centroides toma
	como medida de similaridade entre grupos a distância entre as médias dos
	grupos. Seja o centroide $\mu_i$ de um grupo o seu objeto médio, dados
	pela Equação \ref{eq:centroid}:
	
	\begin{equation}
		\mu_i = \frac{1}{\left|C_i\right|} \sum_{x_i \in C_i}{x_i}
		\label{eq:centroid}
	\end{equation}
	
	Então, a distância entre os grupos é dada pela distância entre os respectivos
	centroides, como mostra a Equação \ref{eq:centroid_distance}:
	
	\begin{equation}
		dist\left(C_i,C_j\right) = \left|\mu_i - \mu_j\right|
		\label{eq:centroid_distance}
	\end{equation}
	
	\item \textbf{Método de Ward}: o método de Ward também considera o centroide
	como representante do grupo todo. Porém, a distância entre grupos é
	medida pela variação da soma dos erros quadrados, ou SSE (\emph{squared sum
	of errors}), que se obtém ao aglomerar dois grupos $C_i$ e $C_j$. Seja a soma
	dos erros quadrados de um grupo $i$, dada pela Equação \ref{eq:sse}:
	
	\begin{equation}
		SSE_i = \sum_{x \in C_i}{{\left| x - \mu_i \right|}^{2}}
		\label{eq:sse}
	\end{equation}
	
	Então, a distância de Ward será dada pela variação da soma dos erros quadrados
	obtida ao aglomerar os grupos $C_i$ e $C_j$:
	
	\begin{equation}
		dist(C_i, C_j) = \Delta SSE_{ij} = SSE_{ij} - SSE_i - SSE_j
		\label{eq:ward}
	\end{equation}
	
	Em \cite{zaki2014data}, é demonstrado que o método de Ward corresponde a uma
	versão ponderada da distância entre centroides:
	
	\begin{equation}
		dist(C_i, C_j) =
			\Delta SSE_{ij} =
			\left( \frac{n_i n_j}{n_i + n_j} \right)
				{\left| \mu_i - \mu_j  \right|}^{2}
		\label{eq:ward_weighted}
	\end{equation}
	
\end{itemize}

A primeira etapa do algoritmo AGNES envolve o cálculo da matriz de distâncias
entre os grupos iniciais. Dado o número $n$ de instâncias no conjunto de dados,
o cálculo da distância entre todos os pares de objetos envolve
$ \frac{n \left(n-1\right)}{2}$ cálculos, considerando
$dist(C_i,C_j) = dist(C_j,C_i)$. Assim, o cálculo da matriz de distâncias
tem complexidade de tempo e espaço $O\left(n^2\right)$.

Na etapa de aglomeração, o laço principal do algoritmo deve ser executado $n-1$
vezes. A cada iteração $i$, a busca pelos pares mais próximos leva tempo 
proporcional a $\left(n-i\right)^2 = O\left(n^2\right)$, assim como o recálculo
da matriz de distâncias. Portanto, o algoritmo AGNES tem complexidade de tempo
$O\left(n^3\right)$ \cite{tan2009introducao}. No entanto, com o emprego de 
estruturas de dados como o \emph{heap} é possível buscar a menor distância em 
tempo $O\left(1\right)$, enquanto que a atualização do \emph{heap} com as novas
distâncias calculadas leva $O\left(\log{n}\right)$. Portanto, a complexidade de
tempo do algoritmo nessas condições é $O\left(n^2 \log{n}\right)$
\cite{zaki2014data}.