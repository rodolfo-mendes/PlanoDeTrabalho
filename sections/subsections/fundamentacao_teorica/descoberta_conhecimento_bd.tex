\subsection{Descoberta de Conhecimento em Bancos de Dados}
	\label{subsec:descoberta_conhecimento_bd}

A Descoberta de Conhecimento em Bancos de Dados, também conhecida pela sigla KDD 
(\emph{Knowledge Discovery in Databases}) é o processo pelo qual dados brutos,
coletados a partir das mais variadas fontes, são processados e transformados em
informações úteis. Por sua vez, estas informações permitem o aprimoramento da
tomada de decisão e até mesmo ampliação do conhecimento científico sobre um
determinado fenômeno \cite{tan2009introducao}.

O processo de KDD envolve desde a aquisição dos dados até a disponibilização do
conhecimento para o usuário final. De acordo com \cite{tan2009introducao},
este processo pode ser descrito pelas seguintes etapas:

\begin{enumerate}
    \item Pré-processamento
    \item mineração de dados
    \item Pós-processamento
\end{enumerate}

O objetivo da etapa de pré-processamento é preparar os dados que alimentarão a 
etapa de mineração de dados. Nesta etapa, podem ser realizadas uma série de
tarefas que visam aumentar a qualidade dos dados fornecidos à mineração de
dados. Na tarefa de \emph{limpeza dos dados}, são tratados atributos sem valor
definido e ruídos. A tarefa de \emph{integração de dados} consiste
em consolidar fontes de dados de diversos tipos (arquivos de texto, planilhas,
\emph{web-services}, arquivos XML, bancos de dados) em uma única fonte de dados
consolidada, usualmente um \emph{data-warehouse}. A \emph{redução da
dimensionalidade} consiste em diminuir o número de atributos que serão
considerados na mineração de dados. Dentre as principais técnicas podemos citar
PCA (\emph{Principal Component Analysis}) e DWT (\emph{Discrete Wavelets
Transforms}). Por fim, a \emph{redução da numerosidade} busca representar o
conjunto de dados através de um número reduzido de instâncias
\cite{han2011data}.

O reconhecimento de padrões é efetivamente realizado na etapa de mineração de
dados. As tarefas desta etapa são categorizadas de acordo com o conhecimento que
se deseja extrair da base de dados analisada. Na tarefa de mineração de itens
frequentes, deseja-se extrair de um banco de transações quais itens ocorrem
conjuntamente com maior frequência. Na tarefa de classificação, o objetivo é
inferir um modelo a partir do qual seja possível prever à qual classe uma
determinada instância de dados pertence. Por fim, na análise de agrupamentos
deseja-se descobrir a existência de grupos (\emph{clusters}) de dados. Assim, é
preciso que se estabeleça uma \emph{medida de similaridade} entre as instâncias
do banco de dados, de forma que se maximize a similaridade entre instâncias do
mesmo grupo e se minimize a similaridade entre instâncias de grupos diferentes.

Por fim, na etapa de pós-processamento avalia-se se os padrões descobertos de
fato representam um \emph{conhecimento} novo sobre os dados. Para cada tipo
de padrão descoberto, pode-se estabelecer uma \emph{medida objetiva} sobre a
qualidade do padrão \cite{han2011data}. No caso dos agrupamentos, por exemplo,
a qualidade destes pode ser medida em termos de \emph{coesão} e \emph{separação}
\cite{tan2009introducao}.

Neste trabalho, será enfatizada a tarefa de agrupamento de dados, com atenção
especial aos algoritmos hierárquicos de agrupamento. Também será discutido como
as técnicas de redução de numerosidade influenciam o tempo de execução dos
algoritmos hierárquicos e a qualidade dos agrupamentos produzidos, como etapa de
pré-processamento.