\subsubsection{DIANA: algoritmo divisivo}
	\label{subsec:diana}

O algoritmo DIANA (DIvisive ANAlysis) é uma abordagem divisiva para o
agrupamento hierárquico \cite{kaufman1990finding}. Ao contrário da abordagem
aglomerativa, na abordagem divisiva o conjunto de dados é tomado como o grupo
inicial do agrupamento. Então, este grupo inicial é sucessivamente dividido em
grupos menores, até que cada instância de dados esteja em seu próprio grupo.
O algoritmo \ref{alg:diana} mostra este procedimento.

\begin{algorithm}[htbp]
	\Entrada{conjunto de dados D}
	\Saida{agrupamento hierárquico}
	
	$\mathcal{C} \gets \left\{ D \right\}$
	
	\Repita{$\left|C\right| = 1, \forall C \in \mathcal{C} $}{
		\ParaCada{$C \in \mathcal{C} \mid \left|C\right| > 1$}{
			$\mathcal{C} \gets \mathcal{C} \cup Divide\left(C\right)$ \;
			$\mathcal{C} \gets \mathcal{C} \setminus \left\{C\right\}$
		}
	}
	
	\caption{DIANA}
	\label{alg:diana}
\end{algorithm}

Devido aos desafios que a abordagem divisiva impõe, esta é pouco estudada na
literatura. Um dos principais problemas encontrados na abordagem divisiva é
justamente a divisão do grupo inicial. Na abordagem aglomerativa, todas as
possíveis fusões são consideradas no passo inicial, dado que, para um conjunto
de dados com $n$ objetos, existem $C_{n}^{2}$ pares possíveis, como mostra a
equação \ref{eq:agglomerative_combinations}:

\begin{equation}
	C_n^2 = \frac{n \left(n-1\right)}{2}
	\label{eq:agglomerative_combinations}
\end{equation}

Porém, para a abordagem divisiva enumerar todas as possíveis divisões de $n$
objetos em dois grupos não vazios, deverão ser consideradas $2^{n-1} - 1$
possibilidades, tornando a análise impraticável até mesmo para pequenos valores
de $n$.

Dada a impossibilidade de se analisar todas as possíveis divisões do grupo
inicial, o algoritmo DIANA utiliza uma heurística simples para dividir os
grupos. Dado um grupo $C_i$ com $n$ objetos, calcula-se a dissimilaridade de
cada objeto $x \in C_i$ através da distância média do objeto $x$ aos objetos
restantes, como mostra a Equação \ref{eq:mean_distance}

\begin{equation}
	\mu_{xi} = \frac{1}{n-1} \sum_{y \in C_i }{\left| x - y \right|}
	\label{eq:mean_distance}
\end{equation}

Para cada grupo que será dividido, mede-se a dissimilaridade de cada
objeto com relação ao resto do grupo. Então, o mais dissimilar é encontrado
e colocado em um grupo separado (\emph{splinter group}). Novamente, as
dissimilaridades são calculadas. Porém, também são calculadas as
dissimilaridades entre os objetos restantes e os objetos do grupo separado. Caso
algum objeto esteja mais próximo do grupo separado, este é transferido. Esse
processo continua até que não haja mais transferências, e ao seu final, obtém-se
a divisão em dois grupos. A Função \ref{func:divide} apresenta este procedimento.

\begin{function}[htbp]

	$ C_j \gets
		\left\{
			x \in C_{ij} \mid 
			\mu_{xij} > \mu_{yij},
			\forall y \neq x
		\right\}
	$ \;
	
	$C_i \gets C_{ij} \setminus C_j $ \;
	
	transferiu $\gets$ \textbf{true}\;
	
	\Enqto{$transferiu = \textbf{true}$}{
		transferiu $\gets$ \textbf{false} \;
		
		\ParaCada{$x \in C_i$}{
			recalcula $\mu_{xi}$ \;
			recalcula $\mu_{xj}$ \;
		}
		
		$ x \gets x \in C_{i} \mid \mu_{xi} > \mu_{yi}, \forall y \neq x$ \;
		
		\Se{$\mu_{xi} > \mu_{xj}$}{
			$C_j \gets C_j \cup \left\{x\right\}$ \;
			$C_i \gets C_i \setminus \left\{x\right\}$ \;
			transferiu $\gets$ \textbf{true} \;
		}
	}
	
	\Retorna{$\left\{C_i,C_j\right\}$}
		
	
	\caption{Divide($C_{ij}$: grupo a ser dividido)}
	\label{func:divide}
\end{function}
