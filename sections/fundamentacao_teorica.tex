\section{Revisão da Literatura Correlata} \label{sec:fundamentacao_teorica}

\subsection{Descoberta de Conhecimento em Bancos de Dados}
	\label{subsec:descoberta_conhecimento_bd}

A Descoberta de Conhecimento em Bancos de Dados, também conhecida pela sigla KDD 
(\emph{Knowledge Discovery in Databases}) é o processo pelo qual dados brutos,
coletados a partir das mais variadas fontes, são processados e transformados em
informações úteis. Por sua vez, estas informações permitem o aprimoramento da
tomada de decisão e até mesmo ampliação do conhecimento científico sobre um
determinado fenômeno \cite{tan2009introducao}.

O processo de KDD envolve desde a aquisição dos dados até a disponibilização do
conhecimento para o usuário final, assim, o processo de KDD pode ser enumerado
nas seguintes etapas:

\begin{enumerate}
    \item Aquisição ou coleta de dados
    \item Pré-processamento
    \item Mineração de dados
    \item Pós-processamento
    \item Apresentação do conhecimento
\end{enumerate}



% introduzir brevemente os tipos de algoritmos de agrupamento, particionamento e hierarquico

% introduzir as tecnicas de agrupamento hierarquico: divisivo versus aglomerativo

% 29/05:
% apresentar os algoritmos single-link, complete-link, birch e dbscan

% explicar por que mesmo que o birch e o dbscan nao podem ser usados?

