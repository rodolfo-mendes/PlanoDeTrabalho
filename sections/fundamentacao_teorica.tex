\section{Revisão da Literatura Correlata} \label{sec:fundamentacao_teorica}

Nesta seção apresentaremos os principais conceitos teóricos relacionados ao
trabalho desenvolvido. Na subseção \ref{subsec:descoberta_conhecimento_bd}
apresentaremos o processo de descoberta de conhecimento em banco de dados e
suas principais etapas. 

\subsection{Descoberta de Conhecimento em Bancos de Dados}
	\label{subsec:descoberta_conhecimento_bd}

A Descoberta de Conhecimento em Bancos de Dados, também conhecida pela sigla KDD 
(\emph{Knowledge Discovery in Databases}) é o processo pelo qual dados brutos,
coletados a partir das mais variadas fontes, são processados e transformados em
informações úteis. Por sua vez, estas informações permitem o aprimoramento da
tomada de decisão e até mesmo ampliação do conhecimento científico sobre um
determinado fenômeno \cite{tan2009introducao}.

O processo de KDD envolve desde a aquisição dos dados até a disponibilização do
conhecimento para o usuário final. De acordo com \cite{tan2009introducao},
este processo pode ser descrito pelas seguintes etapas:

\begin{enumerate}
    \item Pré-processamento
    \item mineração de dados
    \item Pós-processamento
\end{enumerate}

O objetivo da etapa de pré-processamento é preparar os dados que alimentarão a 
etapa de mineração de dados. Nesta etapa, podem ser realizadas uma série de
tarefas que visam aumentar a qualidade dos dados fornecidos à mineração de
dados. Na tarefa de \emph{limpeza dos dados} são tratados atributos sem valor
definido e ruídos. A tarefa de \emph{integração de dados} consiste
em consolidar fontes de dados de diversos tipos (arquivos de texto, planilhas,
\emph{web-services}, arquivos XML, bancos de dados) em uma única fonte de dados
consolidada, usualmente um \emph{data-warehouse}. A \emph{redução da
dimensionalidade} consiste em diminuir o número de atributos que serão
considerados na mineração de dados. Dentre as principais técnicas podemos citar
PCA (\emph{Principal Component Analysis}) e DWT (\emph{Discrete Wavelets
Transforms}). Por fim, a \emph{redução da numerosidade} busca representar o
conjunto de dados através de um número reduzido de instâncias
\cite{han2011data}.

O reconhecimento de padrões é efetivamente realizado na etapa de mineração de
dados. As tarefas desta etapa são categorizadas de acordo com o conhecimento que
se deseja extrair da base de dados analisada. Na tarefa de mineração de itens
frequentes, deseja-se extrair de um banco de transações quais itens ocorrem
conjuntamente com maior frequência. Na tarefa de classificação, o objetivo é
inferir um modelo a partir do qual seja possível prever à qual classe uma
determinada instância de dados pertence. Por fim, na análise de agrupamentos
deseja-se descobrir a existência de grupos (\emph{clusters}) de dados. Assim, é
preciso que se estabeleça uma \emph{medida de similaridade} entre as instâncias
do banco de dados, de forma que se maximize a similaridade entre instâncias do
mesmo grupo e se minimize a similaridade entre instâncias de grupos diferentes.

Por fim, na etapa de pós-processamento avalia-se se os padrões descobertos de
fato representam um \emph{conhecimento} novo sobre os dados. Para cada tipo
de padrão descoberto, pode-se estabelecer uma \emph{medida objetiva} sobre a
qualidade do padrão \cite{han2011data}. No caso dos agrupamentos, por exemplo,
a qualidade destes pode ser medida em termos de \emph{coesão} e \emph{separação}
\cite{tan2009introducao}.

Neste trabalho, será enfatizada a tarefa de agrupamento de dados, com atenção
especial aos algoritmos hierárquicos de agrupamento. Também será discutido como
as técnicas de redução de numerosidade influenciam o tempo de execução dos
algoritmos hierárquicos e a qualidade dos agrupamentos produzidos, como etapa de
pré-processamento.
\subsection{Técnicas de Amostragem de dados}
	\label{subsec:amostragem}
	
O objetivo das técnicas de amostragem de dados é reduzir o número de instâncias
submetidas aos algoritmos de mineração de dados. Entre os desafios da amostragem
de dados estão o balanceamento das instâncias com relação à ocorrência de
instâncias raras ou de exceções. Conside um conjunto $T$ com cardinalidade
$|T| = N$. Entre as técnicas propostas na literatura destacam-se
\cite{Garcia2015}:

\begin{itemize}
    \item Amostragem aleatória de tamanho $s$ sem substituição: criada pela
		escolha de $s$ instâncias de $T$ ($s < N$), onde a probabilidade de um
		exemplo ser escolhido é de $1/N$, de modo que todas as instâncias têm a
		mesma chance de serem escolhidas;
    
    \item Amostragem aleatória de tamanho $s$ com substituição: semelhante à
		anterior, exceto pelo fato que a cada vez que uma instância é escolhida,
		permanece no conjunto e pode ser escolhida novamente;
    
    \item Amostragem balanceada: criada levando-se em consideração um conjunto
		de critérios pré-definidos, por exemplo, para manter a proporcionalidade
		de instâncias entre classes conhecidas;
    
    \item Amostragem de agrupamentos: escolha de grupos específicos resultantes
		de técnicas de agrupamento;
    
    \item Amostragem estratificada: obtida por meio da divisão de um conjunto
		$T$ em partes mutualmente disjunta seguida da escolha de uma amostragem
		aleatória em cada divisão.
\end{itemize}

% Descrever as técnicas correlatas:
% Original Data Squashing (DS) e Likelihood-based Data Squashing (LDS).
\subsection{Análise de Agrupamentos}
	\label{subsec:analise_agrupamentos}
	
A análise de agrupamentos é uma tarefa de mineração de dados cujo objetivo é,
automaticamente, particionar o conjunto de dados em subconjuntos chamados
grupos. Os objetos reunidos em um mesmo grupo devem ser similares entre si,
enquanto que objetos de grupos separados devem ser diferentes. Ao conjunto dos
grupos resultantes da análise dá-se o nome de \emph{agrupamento}.

A análise de agrupamentos pode ser usada como uma ferramenta para extração de
conhecimento sobre um conjunto de dados ou então, como um etapa de
pré-processamento para outras tarefas de mineração de dados. Por exemplo, em
\cite{gonccalves2014land}, a análise de agrupamentos foi utilizada para
identificar o uso do terreno em diferentes regiões do estado de São Paulo,
Brasil. Já em \cite{petitjean2014dynamic}, a análise de agrupamentos foi
utilizada para eleger protótipos que posteriormente seriam utilizados como dados
de treinamento para a tarefa de classificação 1-NN.

Existem diversas abordagens para o agrupamento de dados. No agrupamento por
\emph{particionamento} o conjunto de dados é dividido em $k$ grupos, com cada 
grupo contendo pelo menos um objeto do conjunto. De maneira geral, estes
algoritmos consistem em: a partir de um agrupamento inicial, iterativamente 
realocar os objetos em grupos mais significativos até que um critério de parada
seja atingido. Podemos incluir nesta categoria os algoritmos \emph{k-médias} e
\emph{k-medoids}.

Uma abordagem alternativa é o agrupamento \emph{hierárquico}. Nesta abordagem, 
os objetos são organizados em uma hierarquia de grupos. Por sua vez, esta
hierarquia pode ser construída por duas maneiras diferentes:
\emph{aglomerativa} e \emph{divisiva}.

Na abordagem aglomerativa cada objeto de dados é inicialmente incluído em seu
próprio grupo. Em seguida, cada grupo é aglomerado com o seu grupo mais próximo,
formando uma relação ``pai-filho'' entre o grupo resultante e os grupos menores.
Esse processo se repete até que um único grupo, que contenha todos os dados do 
conjunto, seja obtido. Já na abordagem divisiva o processo se inverte. Todos os
objetos de dados são agrupados em um único grupo inicial, que será a raiz da 
hierarquia. Por sua vez, este grupo inicial é sucessivamente dividido em grupos
menores, até que cada objeto esteja em seu próprio grupo.

Na seção \ref{subsec:abordagem_hierarquica} a abordagem hierárquica será
explorada com mais detalhes.
\subsection{Abordagem Hierárquica para Agrupamentos}
	\label{subsec:abordagem_hierarquica}
	
Na abordagem hierárquica de agrupamento, os grupos são organizados em árvore, de
forma que cada nodo desta árvore representa um grupo. Desta maneira,
estabelece-se uma relação pai-filho entre os grupos, tal que, dado um grupo pai
$C_p$ que tenha $n$ filhos $\{C_1, C_2,...,C_n\}$, então $C_i \subset C_p$ para
todo $1 \leq i \leq n$, como mostra a Figura \ref{fig:cluster_hierarquico}(b).
Nas folhas desta árvore, encontram-se as instâncias de dados, cada uma incluída
em seu próprio grupo. Por sua vez, na raiz encontra-se o grupo que abrange todo
o conjunto de dados. A Figura \ref{fig:cluster_hierarquico}(a) mostra a
hierarquia entre os grupos através de um diagrama conhecido como
\emph{dendrograma}.

\figura{images/cluster_hierarquico.png}{1.0}
{Exemplo de agrupamento hierárquico \cite{tan2009introducao}.}
{fig:cluster_hierarquico}

A organização de grupos em hierarquias tem aplicações em diversas áreas. Em 
recuperação da informação por exemplo, documentos podem ser agrupados de maneira
hierárquica de acordo com o assunto, revelando uma estrutura de tópicos e
sub-tópicos. Em Biologia, o agrupamento de espécies de acordo com suas
características pode ajudar na compreensão sobre como essas espécies evoluíram
ao longo do tempo.

Uma das principais vantagens da abordagem hierárquica de agrupamento é que não
é necessário que o usuário informe o número $k$ de grupos desejado previamente.
Porém, nem sempre é interessante para o usuário analisar todos os grupos obtidos
pelo agrupamento hierárquico. Nestes casos, a hierarquia original pode ser
convertida em um particionamento através do procedimento de poda
(\emph{prunning}). 

Nesta seção serão discutidos os principais aspectos relativos ao agrupamento
hierárquicos. Serão abordados: algoritmo AGNES para agrupamento aglomerativo, 
algoritmo DIANA para agrupamento hierárquico divisivo e as principais medidas
de distância entre clusters: single linkage, complete linkage, average linkage,
mean distance e Ward's distance. Por fim serão analisados vantagens e
desvantagens da abordagem hierárquica.

\subsubsection{AGNES: algoritmo aglomerativo}
	\label{subsec:agnes}

O algoritmo AGNES (\emph{AGglomerative NESting}) é o algoritmo elementar para
executar o agrupamento hierárquico aglomerativo. Basicamente, seu procedimento
consiste alocar inicialmente cada instância de dados em seu próprio grupo e
então, sucessivamente, fundir os grupos mais próximos entre si, até que todos os
objetos sejam aglomerados em um único grupo, como mostra o Algoritmo
\ref{alg:agnes}.

\begin{algorithm}[htbp]
	\Entrada{conjunto de dados $X$}
	\Saida{agrupamento hierárquico}
	
	$n \leftarrow \left|X\right|$ \;
			
	$\mathcal{C} \leftarrow \left\{ C_i = \left\{x_i\right\} | 
		x_i \in X \right\} $ \;
		
	$M \leftarrow \left(m_{ij}\right)_{n \times n}\ |\ m_{ij} = 
		dist\left(C_i,C_j\right), C_i, C_j \in \mathcal{C}$ \;
	
	\Repita{$C \equiv X$}{
		encontra o par de grupos mais próximos $C_i$ e $C_j$ \;
		$C \leftarrow C_i \cup C_j$ \;
		$\mathcal{C} \leftarrow \left( \mathcal{C} \setminus \left\{C_i,C_j\right\}
			\right) \cup \left\{C\right\}$ \;
		recalcula $M$ \;
	}
	
	\caption{Algoritmo aglomerativo para agrupamento hierárquico}
	\label{alg:agnes}
\end{algorithm}


Após cada instância de dados ser atribuída ao seu próprio grupo, a matriz de
distâncias $M$ é calculada, armazenando a distância de cada par de grupos
existente. Então, sucessivamente, o algoritmo localiza a menor entrada na matriz
de distância $M$, que equivale a encontrar o par dos grupos mais próximos,
aglomera os grupos encontrados e recalcula a matriz de distância $M$.

Dado o algoritmo básico do agrupamento hierárquico aglomerativo, a principal
diferença entre as diferentes abordagens nesta categoria são as medidas de 
similaridade entre grupos usadas. Por sua vez, estas podem ser baseadas em
grafos ou baseadas em protótipos \cite{tan2009introducao}. São medidas de
distância baseadas em grafos:

\begin{itemize}
	\item \textbf{Ligação simples}: A ligação simples, ou \emph{single linkage},
	toma como similaridade entre dois grupos a distância entre seus elementos 
	mais próximos, e é calculada pela Equação \ref{eq:single_linkage}:
	
	\begin{equation}
		dist\left(C_i,C_j\right) =
			\min_{x_i \in C_i , x_j \in C_j}
				{\left\{ \left| x_i - x_j \right| \right\}}
		\label{eq:single_linkage}
	\end{equation}
	
	
	Ao tomar-se as instâncias de dados como vértices de um grafo, e as ligações 
	entre grupos como vértices ponderados, então o agrupamento gerado é
	correspondente a uma \emph{árvore geradora mínima} \cite{han2011data}, de
	forma que os grupos formados tendem a ser contíguos no espaço dos atributos
	\cite{tan2009introducao}.
	
	\item \textbf{Ligação completa}: A ligação completa, ou
	\emph{complete linkage} é a medida oposta à ligação simples, pois toma como 
	similaridade entre dois grupos a distância entre seus elementos mais
	distantes, sendo calculada pela Equação \ref{eq:complete_linkage}:
	
	\begin{equation}
		dist\left(C_i,C_j\right) =
			\max_{x_i \in C_i , x_j \in C_j}
				{\left\{ \left| x_i - x_j \right| \right\}}
		\label{eq:complete_linkage}
	\end{equation}
	
	
	\item \textbf{Distância Média}: Por fim, a similaridade entre dois grupos pode
	ser medida através da distância média entre os pares dos grupos
	(\emph{group average}). Essa medida é um balanceamento entre a ligação simples
	e a ligação completa, e é obtida pela média das distâncias entre cada um dos
	pares ordenados $(x_i,x_i)$, com $x_i \in C_i$ e $x_i \in C_j$. É obtida
	pela Equação \ref{eq:group_average}:
	
	\begin{equation}
		dist\left(C_i,C_j\right) =
			\frac{1}{n_in_j} 
			\sum_{x_i \in C_i}{
				\sum_{x_j \in C_j}{
					\left| x_i - x_j \right|
				}
			}
		\label{eq:group_average}
	\end{equation}
	
\end{itemize}

Além das medidas baseadas em grafos, também podem ser aplicadas medidas baseadas
em protótipos. São elas:

\begin{itemize}
	\item \textbf{Distância entre centroides}: a distância entre centroides toma
	como medida de similaridade entre grupos a distância entre as médias dos
	grupos. Seja o centroide $\mu_i$ de um grupo o seu objeto médio, dados
	pela Equação \ref{eq:centroid}:
	
	\begin{equation}
		\mu_i = \frac{1}{\left|C_i\right|} \sum_{x_i \in C_i}{x_i}
		\label{eq:centroid}
	\end{equation}
	
	Então, a distância entre os grupos é dada pela distância entre os respectivos
	centroides, como mostra a Equação \ref{eq:centroid_distance}:
	
	\begin{equation}
		dist\left(C_i,C_j\right) = \left|\mu_i - \mu_j\right|
		\label{eq:centroid_distance}
	\end{equation}
	
	\item \textbf{Método de Ward}: o método de Ward também considera o centroide
	como representante do grupo todo. Porém, a distância entre grupos é
	medida pela variação da soma dos erros quadrados, ou SSE (\emph{squared sum
	of errors}), que se obtém ao aglomerar dois grupos $C_i$ e $C_j$. Seja a soma
	dos erros quadrados de um grupo $i$, dada pela Equação \ref{eq:sse}:
	
	\begin{equation}
		SSE_i = \sum_{x \in C_i}{{\left| x - \mu_i \right|}^{2}}
		\label{eq:sse}
	\end{equation}
	
	Então, a distância de Ward será dada pela variação da soma dos erros quadrados
	obtida ao aglomerar os grupos $C_i$ e $C_j$:
	
	\begin{equation}
		dist(C_i, C_j) = \Delta SSE_{ij} = SSE_{ij} - SSE_i - SSE_j
		\label{eq:ward}
	\end{equation}
	
	Em \cite{zaki2014data}, é demonstrado que o método de Ward corresponde a uma
	versão ponderada da distância entre centroides:
	
	\begin{equation}
		dist(C_i, C_j) =
			\Delta SSE_{ij} =
			\left( \frac{n_i n_j}{n_i + n_j} \right)
				{\left| \mu_i - \mu_j  \right|}^{2}
		\label{eq:ward_weighted}
	\end{equation}
	
\end{itemize}

A primeira etapa do algoritmo AGNES envolve o cálculo da matriz de distâncias
entre os grupos iniciais. Dado o número $n$ de instâncias no conjunto de dados,
o cálculo da distância entre todos os pares de objetos envolve
$ \frac{n \left(n-1\right)}{2}$ cálculos, considerando
$dist(C_i,C_j) = dist(C_j,C_i)$. Assim, o cálculo da matriz de distâncias
tem complexidade de tempo e espaço $O\left(n^2\right)$.

Na etapa de aglomeração, o laço principal do algoritmo deve ser executado $n-1$
vezes. A cada iteração $i$, a busca pelos pares mais próximos leva tempo 
proporcional a $\left(n-i\right)^2 = O\left(n^2\right)$, assim como o recálculo
da matriz de distâncias. Portanto, o algoritmo AGNES tem complexidade de tempo
$O\left(n^3\right)$ \cite{tan2009introducao}. No entanto, com o emprego de 
estruturas de dados como o \emph{heap} é possível buscar a menor distância em 
tempo $O\left(1\right)$, enquanto que a atualização do \emph{heap} com as novas
distâncias calculadas leva $O\left(\log{n}\right)$. Portanto, a complexidade de
tempo do algoritmo nessas condições é $O\left(n^2 \log{n}\right)$
\cite{zaki2014data}.
\subsubsection{DIANA: algoritmo divisivo}
	\label{subsec:diana}

O algoritmo DIANA (DIvisive ANAlysis) é uma abordagem divisiva para o
agrupamento hierárquico \cite{kaufman1990finding}. Ao contrário da abordagem
aglomerativa, na abordagem divisiva o conjunto de dados é tomado como o grupo
inicial do agrupamento. Então, este grupo inicial é sucessivamente dividido em
grupos menores, até que cada instância de dados esteja em seu próprio grupo.
O algoritmo \ref{alg:diana} mostra este procedimento.

\begin{algorithm}[htbp]
	\Entrada{conjunto de dados D}
	\Saida{agrupamento hierárquico}
	
	$\mathcal{C} \gets \left\{ D \right\}$
	
	\Repita{$\left|C\right| = 1, \forall C \in \mathcal{C} $}{
		\ParaCada{$C \in \mathcal{C} \mid \left|C\right| > 1$}{
			$\mathcal{C} \gets \mathcal{C} \cup Divide\left(C\right)$ \;
			$\mathcal{C} \gets \mathcal{C} \setminus \left\{C\right\}$
		}
	}
	
	\caption{DIANA}
	\label{alg:diana}
\end{algorithm}

Devido aos desafios que a abordagem divisiva impõe, esta é pouco estudada na
literatura. Um dos principais problemas encontrados na abordagem divisiva é
justamente a divisão do grupo inicial. Na abordagem aglomerativa, todas as
possíveis fusões são consideradas no passo inicial, dado que, para um conjunto
de dados com $n$ objetos, existem $C_{n}^{2}$ pares possíveis, como mostra a
equação \ref{eq:agglomerative_combinations}:

\begin{equation}
	C_n^2 = \frac{n \left(n-1\right)}{2}
	\label{eq:agglomerative_combinations}
\end{equation}

Porém, para a abordagem divisiva enumerar todas as possíveis divisões de $n$
objetos em dois grupos não vazios, deverão ser consideradas $2^{n-1} - 1$
possibilidades, tornando a análise impraticável até mesmo para pequenos valores
de $n$.

Dada a impossibilidade de se analisar todas as possíveis divisões do grupo
inicial, o algoritmo DIANA utiliza uma heurística simples para dividir os
grupos. Dado um grupo $C_i$ com $n$ objetos, calcula-se a dissimilaridade de
cada objeto $x \in C_i$ através da distância média do objeto $x$ aos objetos
restantes, como mostra a Equação \ref{eq:mean_distance}

\begin{equation}
	\mu_{xi} = \frac{1}{n-1} \sum_{y \in C_i }{\left| x - y \right|}
	\label{eq:mean_distance}
\end{equation}

Para cada grupo que será dividido, mede-se a dissimilaridade de cada
objeto com relação ao resto do grupo. Então, o mais dissimilar é encontrado
e colocado em um grupo separado (\emph{splinter group}). Novamente, as
dissimilaridades são calculadas. Porém, também são calculadas as
dissimilaridades entre os objetos restantes e os objetos do grupo separado. Caso
algum objeto esteja mais próximo do grupo separado, este é transferido. Esse
processo continua até que não haja mais transferências, e ao seu final, obtém-se
a divisão em dois grupos. A Função \ref{func:divide} apresenta este procedimento.

\begin{function}[htbp]

	$ C_j \gets
		\left\{
			x \in C_{ij} \mid 
			\mu_{xij} > \mu_{yij},
			\forall y \neq x
		\right\}
	$ \;
	
	$C_i \gets C_{ij} \setminus C_j $ \;
	
	transferiu $\gets$ \textbf{true}\;
	
	\Enqto{$transferiu = \textbf{true}$}{
		transferiu $\gets$ \textbf{false} \;
		
		\ParaCada{$x \in C_i$}{
			recalcula $\mu_{xi}$ \;
			recalcula $\mu_{xj}$ \;
		}
		
		$ x \gets x \in C_{i} \mid \mu_{xi} > \mu_{yi}, \forall y \neq x$ \;
		
		\Se{$\mu_{xi} > \mu_{xj}$}{
			$C_j \gets C_j \cup \left\{x\right\}$ \;
			$C_i \gets C_i \setminus \left\{x\right\}$ \;
			transferiu $\gets$ \textbf{true} \;
		}
	}
	
	\Retorna{$\left\{C_i,C_j\right\}$}
		
	
	\caption{Divide($C_{ij}$: grupo a ser dividido)}
	\label{func:divide}
\end{function}

\subsection{Fractais e a propriedade de auto-similaridade}
	\label{subsec:fractais}

Um fractal pode ser definido pelo conceito de auto similaridade, no qual partes
de qualquer tamanho de um fractal são similares (exata ou estatisticamente) ao
conjunto todo. Um exemplo clássico de um fractal criado por meio da construção
repetitiva é o triângulo de Sierpinski, construído por meio de um processo
iterativo, onde se retira de um triângulo o triângulo central e para cada
triângulo resultante realiza-se o mesmo processo, recursivamente, conforme
apresentado na Figura \ref{fig:constrsierpinski}. O triângulo de Sierpinski
apresenta características interessantes, como o fato de cada triângulo interior
ser uma miniatura do triângulo em que está inserido, perímetro tendendo ao
infinito e área tendendo a zero quando o número de iterações tende ao infinito
\cite{Schroeder91}. 

\figura{images/sierpinski.png}{0.3}
{Construção do triângulo de Sierpinski \cite{Schroeder91}.}
{fig:constrsierpinski}

O conceito de auto similaridade está relacionado com periodicidade. Muitos
fenômenos naturais e humanos acontecem com periodicidade. São exemplos o uso do
solo pela agricultura, o valor das moedas e das ações, o comportamento de redes
de comunicação, as filas de caixa de supermercado, o uso das estradas, as
músicas que tocam nas rádios e os efeitos da economia na vida das pessoas. Uma
razão para a universalidade dessas movimentações harmônicas é a linearidade
aproximada de muitos sistemas e a sua invariância com deslocamento no espaço e
tempo. Experimentos realizados com alguns conjuntos de dados sintéticos e reais
mostram que os dados referentes aos fenômenos humanos caracterizam-se por
apresentarem uma distribuição fractal \cite{Traina2010}.

Para a análise da auto similaridade de conjuntos contendo fenômenos naturais e
humanos, chamados de fractais estatisticamente auto-similares, utiliza-se o
método \textit{Box-Counting}. Para encontrar o \textit{Box-Counting} de um
conjunto de dados imerso em um espaço $E$-dimensional, deve-se dividir esse
espaço em células de um hipercubo de lado $r$, recursivamente, até encontrar um
elemento por célula \cite{Traina2010}. O método foi proposto para cálculo da
dimensão fractal, que corresponde ao número mínimo de dimensões para
representação de um conjunto (dimensionalidade intrínseca). 

Ainda preciso escrever sobre box plot com base em \cite{Traina2010}.
\subsection{Séries Temporais}
	\label{subsec:series_temporais}

Uma série temporal é uma sequencia ordenada de valores coletados ao longo do
tempo. Elas são usadas para medir fenômenos que variam com o tempo, e sua
análise permite a construção de modelos que explicam estes fenômenos
\cite{morettin2006analise}. São exemplos de séries temporais: o volume de
poupança em um banco ao longo dos meses, a temperatura média de uma cidade
durante a semana, ou ainda, o número de acessos a um \textit{website} durante o dia.

Uma série temporal pode ser representada por uma função $Z(t)$, cujo domínio
representa o tempo, e $t$, um determinado instante no tempo. Dada essa
representação, uma série temporal pode ser classificada em discreta ou contínua,
de acordo com o tipo de conjunto que representa o tempo
\cite{morettin2006analise}. Por exemplo: o faturamento diário de um supermercado
é uma série temporal discreta, pois o tempo é representado por um conjunto de
intervalos enumeráveis, neste caso, dias. Por outro lado, a velocidade de um
carro de corrida durante uma prova é um exemplo de série temporal contínua, pois
a duração da corrida é um intervalo contínuo de tempo. Ainda, uma série temporal
discreta pode ser obtida a partir de uma série contínua, bastando tomar amostras
dos valores da série contínua em intervalos regulares de tempo.

No entanto, há casos em que são necessárias mais de uma variável para
representar o fenômeno medido pela série temporal. Por exemplo, o
\emph{candlestick} diário de uma ação no mercado de capitais é formado por
quatro valores: preço de abertura, preço mínimo, preço máximo e preço de
fechamento. Nestes casos, a série temporal é classificada como multivariada, e é
representada por uma função vetorial
$Z\left(t\right) =
	\left[
		Z_1\left(t\right),
		Z_2\left(t\right),
		...
		Z_r\left(t\right)
	\right]$,
onde r é o número de variáveis que descrevem o fenômeno. Caso $\left(Z_t\right)$
seja um escalar, a série temporal é classificada como univariada
\cite{morettin2006analise}.

% 29/05:
% apresentar os algoritmos single-link, complete-link, birch e dbscan

% explicar por que mesmo que o birch e o dbscan nao podem ser usados?

