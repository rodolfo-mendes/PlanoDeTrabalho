%%%%%%%%%%%%%%%%%%%%%%%%%%%%%%%%%%%%%%%%%%%%%%%%%%%%%%%%%%%%%%%%%%%%%%%%%%%%%%%%
%%Universidade Federal de Uberlândia
%%Faculdade de Computação
%%Programa de Pós-graduação em Ciência da Computação
%%Plano de Trabalho
%%%%%%%%%%%%%%%%%%%%%%%%%%%%%%%%%%%%%%%%%%%%%%%%%%%%%%%%%%%%%%%%%%%%%%%%%%%%%%%%

\documentclass[12pt]{article}
\pagestyle{empty}
\textwidth 16cm \textheight 23.2cm
\voffset -1.5cm \hoffset -1.4cm

\usepackage[utf8]{inputenc}
\usepackage[brazil]{babel}
\usepackage{graphicx,url}
\usepackage{subfigure}
\usepackage{enumitem}
\usepackage{amsfonts}
\usepackage{amsmath}
\usepackage{amsthm}
\usepackage[portuguese,portuguesekw,ruled]{algorithm2e}

\newcommand{\x}{\rule{10mm}{2mm}}

% comando para inserir e formatar uma figura ao texto
% sintaxe:  \figura{arquivo}{escala}{caption}{label}
% exemplo:  \figura{logo.png}{0.5}{Simbolo do MySQL.}{fig:simbolomysql}
\newcommand{\figura}[4] {
    \begin{figure}[htbp]
        \centering
        \includegraphics[scale=#2]{#1}
        \caption{#3}
        \label{#4}
    \end{figure}
}

\DontPrintSemicolon

\newtheorem{definition}{Definição}

\sloppy

\begin{document}

\begin{center}
    \bf{
        \LARGE{PLANO DE TRABALHO}
        \\ $\ $\\
    }
    
    \Large{
        Programa de Mestrado em Ciência da Computação\\
        Universidade Federal de Uberlândia
    }
    \\ $\ $\\
\end{center}

\begin{center}
    \bf{
        Aluno: Rodolfo Martignon Sevilhano Mendes\\ $\ $\\
        Orientador: Prof. Dr. Humberto Luiz Razente\\ $\ $\\
        Título do Trabalho: Uma nova abordagem para aumento da escalabilidade na
				geração de agrupamentos hierárquicos de séries espaço-temporais\\ \ \ \\
        Título do Trabalho: Efeitos da Amostragem no Agrupamento \\
				Hierárquico de Séries Temporais\\ $\ $\\
        Data de Início como Aluno Regular: 11/08/2014\\ $\ $\\
        Previsão da Defesa: Fevereiro de 2017\\ $\ $\\
    }
\end{center}

\section{Introdução} \label{sec:introducao}

A análise de séries temporais de imagens de satélite tem se mostrado uma
importante ferramenta no estudo dos impactos das mudanças climáticas globais.
A partir destas séries temporais, é possível analisar o uso e a cobertura do
solo em uma determinada região, e como este uso varia ao longo do tempo.

Uma das possíveis técnicas para extrair conhecimento destas imagens é o
agrupamento de séries temporais. A partir das imagens de satélite, métricas como
temperatura da superfície, índice de reflexão e índice de vegetação são
extraídos para cada pixel da imagem, para cada instante no tempo. O agrupamento 
das séries temporais obtidas, combinado com a visualização em mapa a quais
grupos cada série pertence, permite ao tomador de decisão entender como o uso do
solo está distribuído em uma região.

Por sua vez, o avanço das tecnologias de sensoreamento têm permitido que
satélites coletem imagens com resoluções cada vez maiores. Isso significa que os
conjuntos de dados são representados por um número cada vez maior de séries
temporais, na ordem de milhões e até mesmo bilhões de séries para serem
agrupadas.

Por meio do algoritmo k-médias é possível agrupar dados com complexidade de
tempo linear, $O(n)$. Porém, esse algoritmo possui algumas desvantagens com
relação ao significado dos agrupamentos gerados. Uma destas desvantagens é a
necessidade do usuário fornecer antecipadamente o número de agrupamentos
desejados através do parâmetro $k$. Ou seja, é necessário que o usuário tenha
algum conhecimento prévio sobre o domínio do problema e que ele tenha uma
estimativa do número de grupos existente no conjunto de dados. Outra limitação
está relacionada ao formato dos agrupamentos no espaço dos atributos.
No k-médias, cada instância do conjunto de dados é atribuída ao grupo cujo
centro esteja mais próximo, que por sua vez é recalculado a cada iteração.
Assim, os agrupamentos tendem a ter formatos esféricos ou regiões densas são
divididas por hiperplanos separadores projetados, o que nem sempre corresponde à
melhor descrição dos dados.

Uma abordagem alternativa são os algoritmos hierárquicos aglomerativos. Nestes
algoritmos, cada instância do conjunto de dados é colocada em seu próprio grupo
inicialmente. Em seguida, os agrupamentos são aglutinados em grupos maiores, 
formando uma hierarquia de grupos. Diferentemente do k-médias, o usuário
não precisa fornecer o número $k$ de grupos previamente. Embora a saída do 
algoritmo hierárquico seja um dendograma (uma hierarquia de grupos), este pode
ser convertido em uma partição de $k$ grupos por meio de uma poda na árvore
resultante. Por sua vez, as partições resultantes deste processo não estão
limitadas a formatos esféricos, sendo o algoritmo capaz de detectar agrupamento
com formatos arbitrários. Apesar dessas vantagens, os algoritmos hierárquicos
tem alto custo computacional, de ordem $O(n^2)$ (quadrática) ou $O(n^3)$
(cúbica) para tempo e espaço, tornando sua aplicação impraticável para grandes
conjuntos de dados como séries temporais de imagens de satélites de alta
resolução.

Neste trabalho, propomos uma nova abordagem para que se possa aproveitar
as vantagens dos algoritmos hierárquicos mesmo em grandes conjuntos de dados.
Esta abordagem consiste em, inicialmente, reduzir o conjunto de dados, e em
seguida, aplicar o agrupamento hierárquico neste conjunto de dados reduzido. Por
fim, as instâncias restantes são atribuídas ao seu vizinho mais próximo (1-NN)
completando o agrupamento. Espera-se que, com esta abordagem, seja possível
aplicar o agrupamento hierárquico em tempo consideravelmente menor, mas sem
perder a qualidade dos agrupamentos resultantes.


\subsection{Objetivos e Desafios de Pesquisa}

O objetivo geral deste trabalho é desenvolver uma abordagem escalável para o
agrupamento hierárquico de séries espaço-temporais de imagens de satélite que
permita reduzir significativamente a complexidade de tempo de execução e espaço
do algoritmo, sem no entanto reduzir a qualidade dos agrupamentos produzidos.
Especificamente, deseja-se:

\begin{enumerate}
    \item Desenvolver uma abordagem para o agrupamento hierárquico aglomerativo
		de séries espaço-temporais incluindo uma etapa de pré-processamento baseada
		em redução de dados, especificamente por meio da geração de amostragens
		aleatória uniforme 
    
    \item Avaliar experimentalmente a nova abordagem em termos de tempo de
		execução, consumo de memória e qualidade dos agrupamentos gerados visando o
		agrupamento dos diversos tipos de vegetação para identificação de áreas de
		plantio de culturas como a cana-de-açúcar
\end{enumerate}

\subsection{Hipótese}

A hipótese deste projeto de pesquisa consiste em: "A redução do conjunto dos
dados de entrada permite executar o agrupamento hierárquico aglomerativo em
tempo menor, mas mantendo a qualidade dos agrupamentos produzidos".

\subsection{Contribuição}

A contribuição esperada é uma estratégia para o agrupamento hierárquico
aglomerativo que produza respostas mais rápidas para o usuário.
\section{Revisão da Literatura Correlata} \label{sec:fundamentacao_teorica}

Nesta seção apresentaremos os principais conceitos teóricos relacionados ao
trabalho desenvolvido. Na subseção \ref{subsec:descoberta_conhecimento_bd}
apresentaremos o processo de descoberta de conhecimento em banco de dados e
suas principais etapas. 

\subsection{Descoberta de Conhecimento em Bancos de Dados}
	\label{subsec:descoberta_conhecimento_bd}

A Descoberta de Conhecimento em Bancos de Dados, também conhecida pela sigla KDD 
(\emph{Knowledge Discovery in Databases}) é o processo pelo qual dados brutos,
coletados a partir das mais variadas fontes, são processados e transformados em
informações úteis. Por sua vez, estas informações permitem o aprimoramento da
tomada de decisão e até mesmo ampliação do conhecimento científico sobre um
determinado fenômeno \cite{tan2009introducao}.

O processo de KDD envolve desde a aquisição dos dados até a disponibilização do
conhecimento para o usuário final. De acordo com \cite{tan2009introducao},
este processo pode ser descrito pelas seguintes etapas:

\begin{enumerate}
    \item Pré-processamento
    \item Mineração de dados
    \item Pós-processamento
\end{enumerate}

O objetivo da etapa de pré-processamento é preparar os dados que alimentarão a 
etapa de mineração de dados. Nesta etapa, podem ser realizadas uma série de
tarefas que visam aumentar a qualidade dos dados fornecidos à mineração de
dados. Na tarefa de \emph{limpeza dos dados}, são tratados atributos sem valor
definido e ruídos. A tarefa de \emph{integração de dados} consiste
em consolidar fontes de dados de diversos tipos (arquivos de texto, planilhas,
\emph{web-services}, arquivos XML, bancos de dados) em uma única fonte de dados
consolidada, usualmente um \emph{data-warehouse}. A \emph{redução da
dimensionalidade} consiste em diminuir o número de atributos que serão
considerados na mineração de dados. Dentre as principais técnicas podemos citar
PCA (\emph{Principal Component Analysis}) e DWT (\emph{Discrete Wavelets
Transforms}). Por fim, a \emph{redução da numerosidade} busca representar o
conjunto de dados através de um número reduzido de instâncias
\cite{han2011data}.

O reconhecimento de padrões é efetivamente realizado na etapa de mineração de
dados. As tarefas desta etapa são categorizadas de acordo com o conhecimento que
se deseja extrair da base de dados analisada. Na tarefa de mineração de itens
frequentes, deseja-se extrair de um banco de transações quais itens ocorrem
conjuntamente com maior frequência. Na tarefa de classificação, o objetivo é
inferir um modelo a partir do qual seja possível prever à qual classe uma
determinada instância de dados pertence. Por fim, na análise de agrupamentos
deseja-se descobrir a existência de grupos (\emph{clusters}) de dados. Assim, é
preciso que se estabeleça uma \emph{medida de similaridade} entre as instâncias
do banco de dados, de forma que se maximize a similaridade entre instâncias do
mesmo grupo e se minimize a similaridade entre instâncias de grupos diferentes.

Por fim, na etapa de pós-processamento avalia-se se os padrões descobertos de
fato representam um \emph{conhecimento} novo sobre os dados. Para cada tipo
de padrão descoberto, pode-se estabelecer uma \emph{medida objetiva} sobre a
qualidade do padrão \cite{han2011data}. No caso dos agrupamentos, por exemplo,
a qualidade destes pode ser medida em termos de \emph{coesão} e \emph{separação}
\cite{tan2009introducao}.

Neste trabalho, será enfatizada a tarefa de agrupamento de dados, com atenção
especial aos algoritmos hierárquicos de agrupamento. Também será discutido como
as técnicas de redução de numerosidade influenciam o tempo de execução dos
algoritmos hierárquicos e a qualidade dos agrupamentos produzidos.


\subsection{Análise de Agrupamentos}
	\label{subsec:analise_agrupamentos}
	
A análise de agrupamentos é uma tarefa de mineração de dados cujo objetivo é,
automaticamente, particionar o conjunto de dados em subconjuntos chamados
grupos. Os objetos reunidos em um mesmo grupo devem ser similares entre si,
enquanto que objetos de grupos separados devem ser diferentes. Ao conjunto dos
grupos resultantes da análise dá-se o nome de \emph{agrupamento}.

A análise de agrupamentos pode ser usada como uma ferramenta para extração de
conhecimento sobre um conjunto de dados ou então, como um etapa de
pré-processamento para outras tarefas de mineração de dados. Por exemplo, em
\cite{gonccalves2014land}, a análise de agrupamentos foi utilizada para
identificar o uso do terreno em diferentes regiões do estado de São Paulo,
Brasil. Já em \cite{petitjean2014dynamic}, a análise de agrupamentos foi
utilizada para eleger protótipos que posteriormente seriam utilizados como dados
de treinamento para a tarefa de classificação 1-NN.

Existem diversas abordagens para o agrupamento de dados. No agrupamento por
\emph{particionamento} o conjunto de dados é dividido em $k$ grupos, com cada 
grupo contendo pelo menos um objeto do conjunto. De maneira geral, estes
algoritmos consistem em: a partir de um agrupamento inicial, iterativamente 
realocar os objetos em grupos mais significativos até que um critério de parada
seja atingido. Podemos incluir nesta categoria os algoritmos \emph{k-médias} e
\emph{k-medoids}.

Uma abordagem alternativa é o agrupamento \emph{hierárquico}. Nesta abordagem, 
os objetos são organizados em uma hierarquia de grupos. Por sua vez, esta
hierarquia pode ser construída por duas maneiras diferentes:
\emph{aglomerativa} e \emph{divisiva}.

Na abordagem aglomerativa cada objeto de dados é inicialmente incluído em seu
próprio grupo. Em seguida, cada grupo é aglomerado com o seu grupo mais próximo,
formando uma relação "pai-filho" entre o grupo resultante e os grupos menores.
Esse processo se repete até que um único grupo, que contenha todos os dados do 
conjunto, seja obtido. Já na abordagem divisiva o processo se inverte. Todos os
objetos de dados são agrupados em um único grupo inicial, que será a raiz da 
hierarquia. Por sua vez, este grupo inicial é sucessivamente dividido em grupos
menores, até que cada objeto esteja em seu próprio grupo.

Na subseção \ref{subsec:abordagem_hierarquica} será a abordagem hierárquica será
explorada com mais detalhes. 

\subsection{Abordagem Hierárquica para Agrupamentos}
	\label{subsec:abordagem_hierarquica}

% introduzir brevemente os tipos de algoritmos de agrupamento, particionamento e
% hierarquico

% introduzir as tecnicas de agrupamento hierarquico: divisivo versus aglomerativo

% 29/05:
% apresentar os algoritmos single-link, complete-link, birch e dbscan

% explicar por que mesmo que o birch e o dbscan nao podem ser usados?


\section{Agrupamento hierárquico aglomerativo de séries espaço-temporais}
	\label{sec:agrupamento_hierarquico}

Nesta seção será descrita a técnica proposta para agrupamento hierárquico de
séries-temporais. Basicamente, esta técnica consiste em reduzir o tamanho do
conjunto de dados por meio de técnicas de amostragem, aplicar o agrupamento
aglomerativo, e finalmente, atribuir as instâncias restantes aos seus grupos
mais próximos, como mostra o Algoritmo \ref{alg:proposta}.

\begin{algorithm}
	\Entrada{Conjunto de dados $D$, número de grupos $K$}
	\Saida{Agrupamento $\mathcal{C}$}
	1. Selecionar amostra $\mathcal{D}$,
		tal que $\left|\mathcal{D}\right| < \left|D\right|$ \;
		
	2. Obter o agrupamento hierárquico
		$\mathcal{H} \gets AGNES\left(\mathcal{D}\right)$ \;
		
	3. Aplicar o procedimento de poda em $\mathcal{H}$, obtendo
		o agrupamento $\mathcal{C}_0$, com $K$ grupos \;
		
	4. Atribuir os objetos restantes em $D \setminus \mathcal{D}$
		aos grupos em $\mathcal{C}_0$, obtendo o agrupamento final $\mathcal{C}$ \;
	
	\caption{Agrupamento Hierárquico Aglomerativo com Amostragem}
	\label{alg:proposta}
\end{algorithm}

A redução de dados tem papel fundamental na obtenção da escalabilidade dos
algoritmos de agrupamento. Assim, o verdadeiro desafio na aplicação das técnicas
de redução de dados é manter as características do conjunto de dados original,
para que seja possível descobrir a estrutura dos grupos presentes.

A redução do tamanho do conjunto de dados por meio de amostragem aleatória é uma
técnica utilizada em vários algoritmos de agrupamento, entre eles CURE
\cite{guha1998cure}, CLARA \cite{kaufman1990finding} e YADING \cite{Ding2015}.
Nestes trabalhos, a amostragem aleatória é aplicada estimando-se o tamanho
mínimo das amostras para que as características dos grupos sejam preservadas.

\subsection{Amostragem Baseada em Fractais}
	\label{subsec:reducao_fractais}

Em vários trabalhos da literatura científica da área a análise de fractais
mostrou-se uma técnica promissora na redução da dimensionalidade dos dados.
No entanto, até o momento, não foram encontrados trabalhos que tenham aplicado
técnicas de análise de fractais à redução de conjuntos de dados
espaço-temporais.

Neste trabalho propõe-se a aplicação da técnica de \emph{Box-plot counting} para
efetuar a redução de dados. Esta técnica permite detectar a auto-similaridade
dos dados pela contagem das células que contêm um ou mais pontos. Essa
característica será usada para a detecção de ruídos e \emph{outliers},
permitindo a escolha de amostras estratificadas pela densidade dos 
hiper-retângulos em vários níveis de resolução.
\section{Métodos de Pesquisa}
	\label{sec:metodologia}
	
O algoritmo proposto na seção \ref{sec:agrupamento_hierarquico} foi implementado em linguagem R\footnote{https://www.r-project.org} e C/C++, de modo que podem
ser usadas diferentes técnicas de amostragem e diferentes medidas de distância, entre elas a distância Euclidiana e a \textit{Dynamic Time Warping}.
O algoritmo proposto vem sendo avaliado em termos do balanceamento do tempo de execução e da qualidade dos agrupamentos 
gerados. O algoritmo será avaliado com diferentes tamanhos de amostras, número de grupos e medidas de distância. Para cada número de grupos e medida de distância, a versão do agrupamento hierárquico com amostragem será comparada ao agrupamento hierárquico convencional, isto é, com todos os dados do conjunto e também com o agrupamento gerado aleatoriamente.

\subsection{Conjuntos de Dados}

Os conjuntos de dados utilizados nos experimentos foram processados e fornecidos pela EMBRAPA \cite{embrapa2016} e têm resolução de 1 km/pixel. Na próxima etapa dos experimentos serão utilizados conjuntos de dados provenientes de imagens com resolução de 250 m/pixel. Outros conjuntos disponíveis  no repositório em \cite{global2016} também serão avaliados.

% Os dados sobre os quais os algoritmos serão avaliados serão coletados pela EMBRAPA e do repositório em \cite{global2016}.


\section{Resultados Preliminares}
	\label{sec:resultados_preliminares}

Nesta seção serão descritos os experimentos preliminares cujo intuito foi
comparar o agrupamento hierárquico por amostragem, proposto na seção
\ref{sec:agrupamento_hierarquico} e o algoritmo AGNES, que constrói o
agrupamento hierárquico aglomerativo com todo o conjunto de dados. Os resultados
tabelados podem ser encontrados no Apêndice.

O conjunto de dados utilizado nos testes são valores de índice NDVI
(\emph{Normalized Difference Vegetation Index}) de uma área localizada entre
as latitudes $-8,55$ e $-8,45$, e as longitudes $-38,25$ e $37,25$, que 
corresponde a uma região do estado de Pernambuco, Brasil. Estes dados foram
extraídos a partir de imagens coletadas por um satélite MODIS
(\emph{Moderate Resolution Imaging Spetroradiometer}) durante o ano de 2003,
em períodos de 16 dias. Assim, a cada \emph{pixel} da imagem é associada uma
série temporal composta pelos valores do índice NDVI coletados ao longo do ano.
O conjunto de dados utilizado possui um total de $9812$ séries temporais, cada
uma $23$ coletas do índice NDVI. A tabela \ref{tab:amostra_dados} mostra um
subconjunto destes dados.

% latex table generated in R 3.2.4 by xtable 1.8-2 package
% Mon Jun 13 22:43:15 2016
\begin{table}[htbp]
\centering
\begin{tabular}{|c|c|r|r|r|}
  \hline
	\textbf{Latitude} & \textbf{Longitude} & \textbf{01/01/2003} & \textbf{17/01/2003} & \textbf{02/02/2003} \\ 
  \hhline{|=|=|=|=|=|}
	-8.36 & -40.89 & 0.76 & 0.83 & 0.76 \\ \hline
  -8.36 & -40.89 & 0.71 & 0.83 & 0.76 \\ \hline
  -8.36 & -40.89 & 0.73 & 0.81 & 0.62 \\ \hline
  -8.36 & -40.89 & 0.68 & 0.74 & 0.72 \\ \hline
  -8.36 & -40.89 & 0.69 & 0.74 & 0.72 \\
	\hline
\end{tabular}
\caption{Amostras do conjunto de dados MODIS-PE-2003.} 
\label{tab:amostra_dados}
\end{table}


O experimento consistiu em comparar o algoritmo hierárquico aglomerativo por
amostragem com o algoritmo AGNES, ambos utilizando a ligação simples
(\emph{single linkage}) como medida de distância entre grupos. Os algoritmos
foram comparados em relação ao seu tempo de execução e à qualidade dos
agrupamentos gerados. Para medir a qualidade dos agrupamentos, utilizou-se
o índice Dunn e o índice Davies-Bouldin.

Para cada um dos algoritmos comparados, foi realizada a poda da hierarquia
gerada em grupos $k = 3,5,7$. Por ser determinístico, o algoritmo AGNES foi
executado uma única vez para cada valor de $k$. Já o algoritmo por amostragem
foi executado com amostras de tamanho $m = 10,100,1000$. Por sua natureza 
probabilística, o algoritmo por amostragem foi executado $10$ vezes para cada
combinação de $m$ e $k$, e foram calculadas as médias do tempo de execução
e das métricas de qualidade dos agrupamentos. O gráfico da Figura 
\ref{fig:tempo_execucao} mostra a comparação dos tempos de execução em escala
logarítmica (a amostra de tamanho 9812 corresponde à execução do algoritmo
AGNES).

\figura{images/executionTime.pdf}{0.6}
{Tempo de execução por tamanho de amostra}
{fig:tempo_execucao}

Embora o algoritmo baseado em amostragem precise, ao final do algoritmo,
atribuir as instâncias restantes ao grupo mais próximo, o tempo de execução 
dessa etapa é da ordem de $O\left(nm\right)$, ou seja, linear em relação ao 
tamanho $n$ do conjunto de dados. Assim, diminuir o número de instâncias
processadas pelo algoritmo aglomerativo representou uma diminuição significativa
no tempo de execução do agrupamento, como esperado. 

Para comparar a qualidade dos agrupamentos produzidos foram utilizados os
índices Dunn e Davies-Boulding. Além do tamanho da amostragem, também foi
avaliado se o número de grupos utilizados na poda influenciaria a qualidade dos
agrupamentos produzidos.

O índice Dunn é obtido pela razão entre a menor distância entre pares de grupos
diferentes e a maior distância entre pares de um mesmo grupo. Assim, para
agrupamentos de maior qualidade, o valor desse índice é maior. Como pode ser 
observado no gráfico da Figura \ref{fig:indice_dunn}, o número de grupos
utilizado na poda influenciou pouco a qualidade dos agrupamentos obtidos. Por
sua vez, o fator determinante na qualidade foi o tamanho da amostra utilizada
nos algoritmos.

\figura{images/indiceDunn.pdf}{0.6}
{Índice Dunn por tamanho da amostra e número de grupos}
{fig:indice_dunn}

Houve grande diferença entre a qualidade dos agrupamentos produzidos pelo
algoritmo AGNES e os agrupamentos produzidos pelo algoritmo baseado em
amostragem. Dado que o índice Dunn é influenciado pela menor distância entre
pares de grupos diferentes, esse desempenho pode ser explicado pelo fato do 
algoritmo AGNES garantir que os pares de objetos mais próximos serão colocados 
no mesmo grupo. Por outro lado, no algoritmo baseado em amostragem, existe a
possibilidade de que objetos muito próximos sejam colocados em grupos separados,
afetando negativamente seu desempenho.

A qualidade inferior dos agrupamentos produzidos pelo algoritmo baseado em 
amostragem também foi refletida no índice Davies-Boulding. Este índice é obtido
para cada par de grupos como a relação entre a soma dos desvios-padrão e a 
distância entre as médias dos grupos. Assim, quanto maior a qualidade dos 
agrupamentos, menor será o valor desse índice, pois menor será a dispersão
dentro de um grupo e maior será a distância entre seus centros. O gráfico da
Figura \ref{tab:indice_db} mostra os resultados obtidos.

\figura{images/indiceDaviesBouldin.pdf}{0.6}
{Índice Davies-Bouldin por K e tamanho de amostra}
{fig:indice_davies_bouldin}

Os dados mostrados no gráfico da Figura \ref{tab:indice_db} também indicam que a
qualidade dos agrupamentos produzidos pelo algoritmo baseado em amostragem foi
inferior. Isso também pode ser explicado pela garantia que o algoritmo AGNES
oferece de agrupar os pares mais próximos já no primeiro passo do algoritmo,
pois diminui a dispersão dos dados. O gráfico da Figura
\ref{fig:indice_davies_bouldin} mostra que, com relação ao índice Dunn, este
índice sofreu maior variação de acordo com o número de grupos e o tamanho das
amostras.

\subsection{Conclusões Preliminares}
	\label{subsec:conclusoes_preliminares}
	
Por meio destes experimentos confirmou-se que o algoritmo proposto na Seção
\ref{sec:agrupamento_hierarquico} permite que o agrupamento hierárquico
aglomerativo seja executado em tempo muito menor que o algoritmo original.
Porém, a amostragem aleatória impactou negativamente na qualidade dos
agrupamentos obtidos.

No entanto, ainda não é possível descartar a utilização das técnicas de redução
de dados para obter melhor desempenho do agrupamento hierárquico aglomerativo.
Ao longo do trabalho proposto, pretende-se ampliar o escopo dos experimentos,
averiguando o comportamento de outras medidas de distância, variar o tamanho
das amostras e o número de grupos, assim como implementar novos métodos de 
redução de dados.

\section{Cronograma de Execução}
	\label{sec:cronograma}

Para explorar a metodologia utilizada as atividades a serem realizadas foram
definidas e detalhadas segundo seu período de execução, conforme a Tabela
\ref{tab:cronograma}.

\begin{enumerate}
\item Finalização da implementação do método de amostragem baseado em fractais;
\item Realização de experimentos adicionais, comparando diferentes tipos de 
		  amostragem, conjuntos de dados e parâmetros;
\item Redação de artigo para publicação dos resultados;
\item Preparação da dissertação;
\item Defesa da dissertação.    
\end{enumerate}  

\begin{table}[htbp]
\centering
\setlength{\tabcolsep}{0pt}
\begin{tabular}{|c|c|c|c|c|c|c|c|c|c|c|c|c|c|c|}
\hline
\multirow{2}{*}{ \ Atividades \ } & \multicolumn{6}{c|}{2015}               & \multicolumn{8}{c|}{2016} \\ \cline{2-15} 
                                  & Jul. & Ago. & Set. & Out. & Nov. & Dez. & Jan. & Fev. & Mar. & Abr. & Mai. & Jun. & Jul. & Ago. \\ \hline
1                                 & \x   & \x   & \x   &      &      &      &      &      &      &      &      &      &      &      \\ \hline
2                                 & \x   & \x   & \x   &      &      &      &      &      &      &      &      &      &      &     	\\ \hline
3                                 &      &      &      & \x   & \x   &      &      &      &      &      &      &      &      &     	\\ \hline
4                                 &      &      &      &      &      & \x   & \x   &      &      &      &      &      &      &     	\\ \hline
5                                 &      &      &      &      &      &      &      & \x   & \x   & \x   &      &      &      &     	\\ \hline
6                                 &      &      &      &      &      &      &      &      &      &      & \x   & \x   & \x   &     	\\ \hline
7                                 &      &      &      &      &      &      &      &      &      &      &      &      &      & \x  	\\ \hline
\end{tabular}
\caption{Cronograma de execução}
\label{tab:cronograma}
\end{table}

\ \\
\noindent Uberlândia, 14 de Dezembro de 2015.\\

\ \\
\noindent \textbf{Assinatura do Orientador:} \\

\ \\
\noindent \textbf{Assinatura do Aluno:}
\bibliographystyle{sbc}
\bibliography{bib/plano,bib/books}
\end{document}